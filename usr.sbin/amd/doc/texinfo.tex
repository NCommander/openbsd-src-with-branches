%% TeX macros to handle texinfo files

%   Copyright (C) 1985, 1986, 1988 Free Software Foundation, Inc.

%GNU CC is free software; you can redistribute it and/or modify
%it under the terms of the GNU General Public License as published by
%the Free Software Foundation; either version 1, or (at your option)
%any later version.

%GNU CC is distributed in the hope that it will be useful,
%but WITHOUT ANY WARRANTY; without even the implied warranty of
%MERCHANTABILITY or FITNESS FOR A PARTICULAR PURPOSE.  See the
%GNU General Public License for more details.

%You should have received a copy of the GNU General Public License
%along with GNU CC; see the file COPYING.  If not, write to
%the Free Software Foundation, 675 Mass Ave, Cambridge, MA 02139, USA.


%In other words, you are welcome to use, share and improve this program.
%You are forbidden to forbid anyone else to use, share and improve
%what you give them.   Help stamp out software-hoarding!

\def\texinfoversion{1.26}
\message{Loading texinfo package [Version \texinfoversion]:}
\message{}

% Save some parts of plain tex whose names we will redefine.

\let\ptexlbrace=\{
\let\ptexrbrace=\}
\let\ptexdot=\.
\let\ptexstar=\*
\let\ptexend=\end
\let\ptexbullet=\bullet
\let\ptexb=\b
\let\ptexc=\c
\let\ptexi=\i
\let\ptext=\t
\let\ptexl=\l
\let\ptexL=\L

\def\tie{\penalty 10000\ }     % Save plain tex definition of ~.

\message{Basics,}
\chardef\other=12

\hyphenation{ap-pen-dix}
\hyphenation{mini-buf-fer mini-buf-fers}
\hyphenation{eshell}

% Margin to add to right of even pages, to left of odd pages.
\newdimen \bindingoffset  \bindingoffset=0pt
\newdimen \normaloffset   \normaloffset=\hoffset
\newdimen\pagewidth \newdimen\pageheight
\pagewidth=\hsize \pageheight=\vsize

%---------------------Begin change-----------------------
%
% Dimensions to add cropmarks at corners Added by P. A. MacKay, 12 Nov. 1986
%
\newdimen\cornerlong \newdimen\cornerthick
\newdimen \topandbottommargin
\newdimen \outerhsize \newdimen \outervsize
\cornerlong=1pc\cornerthick=.3pt	% These set size of cropmarks
\outerhsize=7in
\outervsize=9.5in
\topandbottommargin=.75in
%
%---------------------End change-----------------------

% \onepageout takes a vbox as an argument.  Note that \pagecontents
% does insertions itself, but you have to call it yourself.
\chardef\PAGE=255  \output={\onepageout{\pagecontents\PAGE}}
\def\onepageout#1{\hoffset=\normaloffset
\ifodd\pageno  \advance\hoffset by \bindingoffset
\else \advance\hoffset by -\bindingoffset\fi
\shipout\vbox{{\let\hsize=\pagewidth \makeheadline} \pagebody{#1}%
 {\let\hsize=\pagewidth \makefootline}}
\advancepageno \ifnum\outputpenalty>-20000 \else\dosupereject\fi}


% Here is a modification of the main output routine for Near East Publications
% This provides right-angle cropmarks at all four corners.
% The contents of the page are centerlined into the cropmarks,
% and any desired binding offset is added as an \hskip on either
% site of the centerlined box.  (P. A. MacKay, 12 November, 1986)
%
\def\croppageout#1{\hoffset=0pt % make sure this doesn't mess things up
		 \shipout
		 \vbox to \outervsize{\hsize=\outerhsize
                 \vbox{\line{\ewtop\hfill\ewtop}}
                 \nointerlineskip
                 \line{\vbox{\moveleft\cornerthick\nstop}
                       \hfill
                       \vbox{\moveright\cornerthick\nstop}}
                 \vskip \topandbottommargin
                 \centerline{\ifodd\pageno\hskip\bindingoffset\fi
			\vbox{
			{\let\hsize=\pagewidth \makeheadline}
			\pagebody{#1}
			{\let\hsize=\pagewidth \makefootline}}
			\ifodd\pageno\else\hskip\bindingoffset\fi}
		 \vskip \topandbottommargin plus1fill minus1fill
                 \boxmaxdepth\cornerthick
                 \line{\vbox{\moveleft\cornerthick\nsbot}
                       \hfill
                       \vbox{\moveright\cornerthick\nsbot}}
                 \nointerlineskip
                 \vbox{\line{\ewbot\hfill\ewbot}}
	}
  \advancepageno 
  \ifnum\outputpenalty>-20000 \else\dosupereject\fi}
%
% Do @cropmarks to get crop marks
\def\cropmarks{\let\onepageout=\croppageout }

\def\pagebody#1{\vbox to\pageheight{\boxmaxdepth=\maxdepth #1}}
{\catcode`\@ =11
\gdef\pagecontents#1{\ifvoid\topins\else\unvbox\topins\fi
\dimen@=\dp#1 \unvbox#1
\ifvoid\footins\else\vskip\skip\footins\footnoterule \unvbox\footins\fi
\ifr@ggedbottom \kern-\dimen@ \vfil \fi}
}

%
% Here are the rules for the cropmarks.  Note that they are
% offset so that the space between them is truly \outerhsize or \outervsize
% (P. A. MacKay, 12 November, 1986)
%
\def\ewtop{\vrule height\cornerthick depth0pt width\cornerlong}
\def\nstop{\vbox
  {\hrule height\cornerthick depth\cornerlong width\cornerthick}}
\def\ewbot{\vrule height0pt depth\cornerthick width\cornerlong}
\def\nsbot{\vbox
  {\hrule height\cornerlong depth\cornerthick width\cornerthick}}

% Parse an argument, then pass it to #1.
% The argument can be delimited with [...] or with "..." or braces
% or it can be a whole line.
% #1 should be a macro which expects
% an ordinary undelimited TeX argument.

\def\parsearg #1{\let\next=#1\begingroup\obeylines\futurelet\temp\parseargx}

\def\parseargx{%
\ifx \obeyedspace\temp \aftergroup\parseargdiscardspace \else%
\aftergroup \parseargline %
\fi \endgroup}

{\obeyspaces %
\gdef\parseargdiscardspace {\begingroup\obeylines\futurelet\temp\parseargx}}

\gdef\obeyedspace{\ }

\def\parseargline{\begingroup \obeylines \parsearglinex}
{\obeylines %
\gdef\parsearglinex #1^^M{\endgroup \next {#1}}}

\def\flushcr{\ifx\par\lisppar \def\next##1{}\else \let\next=\relax \fi \next}

%% These are used to keep @begin/@end levels from running away
%% Call \inENV within environments (after a \begingroup)
\newif\ifENV \ENVfalse \def\inENV{\ifENV\relax\else\ENVtrue\fi}
\def\ENVcheck{%
\ifENV\errmessage{Still within an environment.  Type Return to continue.}
\endgroup\fi} % This is not perfect, but it should reduce lossage

% @begin foo  is the same as @foo, for now.
\newhelp\EMsimple{Type <Return> to continue}

\outer\def\begin{\parsearg\beginxxx}

\def\beginxxx #1{%
\expandafter\ifx\csname #1\endcsname\relax
{\errhelp=\EMsimple \errmessage{Undefined command @begin #1}}\else
\csname #1\endcsname\fi}

%% @end foo executes the definition of \Efoo.
%% foo can be delimited by doublequotes or brackets.

\def\end{\parsearg\endxxx}

\def\endxxx #1{%
\expandafter\ifx\csname E#1\endcsname\relax
\expandafter\ifx\csname #1\endcsname\relax
\errmessage{Undefined command @end #1}\else
\errorE{#1}\fi\fi
\csname E#1\endcsname}
\def\errorE#1{
{\errhelp=\EMsimple \errmessage{@end #1 not within #1 environment}}}

% Single-spacing is done by various environments.

\newskip\singlespaceskip \singlespaceskip = \baselineskip
\def\singlespace{%
{\advance \baselineskip by -\singlespaceskip
\kern \baselineskip}%
\baselineskip=\singlespaceskip
}

%% Simple single-character @ commands

% @@ prints an @
% Kludge this until the fonts are right (grr).
\def\@{{\sf \char '100}}

% Define @` and @' to be the same as ` and '
% but suppressing ligatures.
\def\`{{`}}
\def\'{{'}}

% Used to generate quoted braces.

\def\mylbrace {{\tt \char '173}}
\def\myrbrace {{\tt \char '175}}
\let\{=\mylbrace
\let\}=\myrbrace

% @: forces normal size whitespace following.
\def\:{\spacefactor=1000 }

% @* forces a line break.
\def\*{\hfil\break}

% @. is an end-of-sentence period.
\def\.{.\spacefactor=3000 }

% @w prevents a word break
\def\w #1{\hbox{#1}}

% @group ... @end group  forces ... to be all on one page.

\def\group{\begingroup% \inENV ???
\def \Egroup{\egroup\endgroup}
\vbox\bgroup}

% @br   forces paragraph break

\let\br = \par

% @dots{}  output some dots

\def\dots{$\ldots$}

% @page    forces the start of a new page

\def\page{\par\vfill\supereject}

% @exdent text....
% outputs text on separate line in roman font, starting at standard page margin

\def\exdent{\errmessage{@exdent in filled text}}
  % @lisp, etc, define \exdent locally from \internalexdent

{\obeyspaces
\gdef\internalexdent{\parsearg\exdentzzz}}

\def\exdentzzz #1{{\advance \leftskip by -\lispnarrowing
\advance \hsize by -\leftskip
\advance \hsize by -\rightskip
\leftline{{\rm#1}}}}

% @include file    insert text of that file as input.

\def\include{\parsearg\includezzz}
\def\includezzz #1{{\def\thisfile{#1}\input #1
}}

\def\thisfile{}

% @center line   outputs that line, centered

\def\center{\parsearg\centerzzz}
\def\centerzzz #1{{\advance\hsize by -\leftskip
\advance\hsize by -\rightskip
\centerline{#1}}}

% @sp n   outputs n lines of vertical space

\def\sp{\parsearg\spxxx}
\def\spxxx #1{\par \vskip #1\baselineskip}

% @comment ...line which is ignored...
% @c is the same as @comment
% @ignore ... @end ignore  is another way to write a comment

\def\comment{\parsearg \commentxxx}

\def\commentxxx #1{}

\let\c=\comment

% Prevent errors for section commands.
% Used in @ignore and in failing conditionals.
\def\ignoresections{%
\let\chapter=\relax
\let\unnumbered=\relax
\let\unnumberedsec=\relax
\let\unnumberedsection=\relax
\let\unnumberedsubsec=\relax
\let\unnumberedsubsection=\relax
\let\unnumberedsubsubsec=\relax
\let\unnumberedsubsubsection=\relax
\let\section=\relax
\let\subsec=\relax
\let\subsubsec=\relax
\let\subsection=\relax
\let\subsubsection=\relax
\let\appendix=\relax
\let\appendixsec=\relax
\let\appendixsection=\relax
\let\appendixsubsec=\relax
\let\appendixsubsection=\relax
\let\appendixsubsubsec=\relax
\let\appendixsubsubsection=\relax
}

\def\ignore{\begingroup\ignoresections\ignorexxx}
\long\def\ignorexxx #1\end ignore{\endgroup}

% Conditionals to test whether a flag is set.

\outer\def\ifset{\begingroup\ignoresections\parsearg\ifsetxxx}

\def\ifsetxxx #1{\endgroup
\expandafter\ifx\csname IF#1\endcsname\relax \let\temp=\ifsetfail
\else \let\temp=\relax \fi
\temp}
\def\Eifset{}
\def\ifsetfail{\begingroup\ignoresections\ifsetfailxxx}
\long\def\ifsetfailxxx #1\end ifset{\endgroup}

\outer\def\ifclear{\begingroup\ignoresections\parsearg\ifclearxxx}

\def\ifclearxxx #1{\endgroup
\expandafter\ifx\csname IF#1\endcsname\relax \let\temp=\relax
\else \let\temp=\ifclearfail \fi
\temp}
\def\Eifclear{}
\def\ifclearfail{\begingroup\ignoresections\ifclearfailxxx}
\long\def\ifclearfailxxx #1\end ifclear{\endgroup}

% Some texinfo constructs that are trivial in tex

\def\iftex{}
\def\Eiftex{}
\def\ifinfo{\begingroup\ignoresections\ifinfoxxx}
\long\def\ifinfoxxx #1\end ifinfo{\endgroup}
\long\def\menu #1\end menu{}
\def\asis#1{#1}

\def\node{\parsearg\nodezzz}
\def\nodezzz#1{\nodexxx [#1,]}
\def\nodexxx[#1,#2]{\gdef\lastnode{#1}}
\let\lastnode=\relax

\def\donoderef{\ifx\lastnode\relax\else
\expandafter\expandafter\expandafter\setref{\lastnode}\fi
\let\lastnode=\relax}

\def\unnumbnoderef{\ifx\lastnode\relax\else
\expandafter\expandafter\expandafter\unnumbsetref{\lastnode}\fi
\let\lastnode=\relax}

\let\refill=\relax
  
% @setfilename is done at the beginning of every texinfo file.
% So open here the files we need to have open while reading the input.
% This makes it possible to make a .fmt file for texinfo.
\def\setfilename{%
   \readauxfile
   \opencontents
   \openindices
   \fixbackslash  % Turn off hack to swallow `\input texinfo'.
   \comment % Ignore the actual filename.
}

\outer\def\bye{\pagealignmacro\tracingstats=1\ptexend}

\def\inforef #1{\inforefzzz #1,,,,**}
\def\inforefzzz #1,#2,#3,#4**{See Info file \file{\losespace#3{}}, node `\losespace#1{}'}
\def\losespace #1{#1}

\message{fonts,}

% Font-change commands.

%% Try out Computer Modern fonts at \magstephalf
\font\tenrm=cmr10 scaled \magstephalf
\font\tentt=cmtt10 scaled \magstephalf
% Instead of cmb10, you many want to use cmbx10.
% cmbx10 is a prettier font on its own, but cmb10
% looks better when embedded in a line with cmr10.
\font\tenbf=cmb10 scaled \magstephalf 
\font\tenit=cmti10 scaled \magstephalf
\font\tensl=cmsl10 scaled \magstephalf
\font\tensf=cmss10 scaled \magstephalf
\def\li{\sf}
\font\tensc=cmcsc10 scaled \magstephalf

% Fonts for @defun, etc.
\font\defbf=cmbx10 scaled \magstep1 %was 1314
\let\deftt=\tentt
\def\df{\let\tt=\deftt \defbf}

% Font for title
\font\titlerm = cmbx10 scaled \magstep5

% Fonts for indices
\font\indit=cmti9 \font\indrm=cmr9
\def\indbf{\indrm} \def\indsl{\indit}
\def\indexfonts{\let\it=\indit \let\sl=\indsl \let\bf=\indbf \let\rm=\indrm}

% Fonts for headings
\font\chaprm=cmbx10 scaled \magstep3
\font\chapit=cmti10 scaled \magstep3
\font\chapsl=cmsl10 scaled \magstep3
\font\chaptt=cmtt10 scaled \magstep3
\font\chapsf=cmss10 scaled \magstep3
\let\chapbf=\chaprm

\font\secrm=cmbx10 scaled \magstep2
\font\secit=cmti10 scaled \magstep2
\font\secsl=cmsl10 scaled \magstep2
\font\sectt=cmtt10 scaled \magstep2
\font\secsf=cmss10 scaled \magstep2
\let\secbf=\secrm

% \font\ssecrm=cmbx10 scaled \magstep1    % This size an fontlooked bad.
% \font\ssecit=cmti10 scaled \magstep1    % The letters were too crowded.
% \font\ssecsl=cmsl10 scaled \magstep1
% \font\ssectt=cmtt10 scaled \magstep1
% \font\ssecsf=cmss10 scaled \magstep1

\font\ssecrm=cmb10 at 13pt	% Note the use of cmb rather than cmbx.
\font\ssecit=cmti10 at 13pt	% Also, the size is a little larger than
\font\ssecsl=cmsl10 at 13pt	% being scaled magstep1.
\font\ssectt=cmtt10 at 13pt
\font\ssecsf=cmss10 at 13pt

\let\ssecbf=\ssecrm

\def\textfonts{\let\rm=\tenrm\let\it=\tenit\let\sl=\tensl\let\bf=\tenbf%
\let\smallcaps=\tensc\let\sf=\tensf}
\def\chapfonts{\let\rm=\chaprm\let\it=\chapit\let\sl=\chapsl\let\bf=\chapbf\let\tt=\chaptt\let\sf=\chapsf}
\def\secfonts{\let\rm=\secrm\let\it=\secit\let\sl=\secsl\let\bf=\secbf\let\tt=\sectt\let\sf=\secsf}
\def\subsecfonts{\let\rm=\ssecrm\let\it=\ssecit\let\sl=\ssecsl\let\bf=\ssecbf\let\tt=\ssectt\let\sf=\ssecsf}
% Count depth in font-changes, for error checks
\newcount\fontdepth \fontdepth=0

% Font for table of contents.
\font\truesecrm=cmr12

%% Add scribe-like font environments, plus @l for inline lisp (usually sans
%% serif) and @ii for TeX italic

\def\i#1{{\sl #1}}
\let\var=\i
\let\dfn=\i
\let\emph=\i
\let\cite=\i

\def\b#1{{\bf #1}}
\let\strong=\b

\def\t#1{{\tt \rawbackslash \frenchspacing #1}\null}
\let\ttfont = \t
\let\kbd=\t
\let\code=\t
\def\samp #1{`{\tt \rawbackslash \frenchspacing #1}'\null}
\def\key #1{{\tt \uppercase{#1}}\null}
\def\ctrl #1{{\tt \rawbackslash \hat}#1}

\let\file=\samp

\def\l#1{{\li #1}\null}		% 

\def\r#1{{\rm #1}}		% roman font
\def\sc#1{{\\smallcaps #1}}	% smallcaps font
\def\ii#1{{\it #1}}		% italic font

\def\titlefont#1{{\titlerm #1}}

% Make altmode in file print out right

\catcode `\^^[=\active \def^^[{$\diamondsuit$}

\message{page headings,}

\newskip\titlepagetopglue \titlepagetopglue = 1.5in
\newskip\titlepagebottomglue \titlepagebottomglue = 2pc

% First the title page.  Must do @settitle before @titlepage.
\font\titlerm = cmbx12 scaled \magstep2
\def\titlefont#1{{\titlerm #1}}

\newtoks\realeverypar
\newif\ifseenauthor

\def\titlepage{\begingroup \parindent=0pt \textfonts
   \font\subtitlerm = cmr10 scaled \magstephalf
   \def\subtitlefont{\subtitlerm \normalbaselineskip = 12pt \normalbaselines}%
   %
   \font\authorrm = cmbx12 scaled \magstep1
   \def\authorfont{\authorrm \normalbaselineskip = 16pt \normalbaselines}%
   %
   % The first subtitle should have some space before it, but not the
   % others.  They all should be ragged left.
% ??? This code turned off because (1) it is wrong for all old title
% pages, and (2) it makes an extra group which never is ended.
%   \begingroup \realeverypar = {\leftskip = 2in plus 3em minus 1em
%                    \parfillskip = 0pt}%
%   \everypar = {\vglue \baselineskip \the\realeverypar
%                \everypar={\the\realeverypar}}%
   %
   % Now you can print the title using @title.
   \def\title{\parsearg\titlezzz}%
   \def\titlezzz##1{\leftline{\titlefont{##1}
		    \vskip4pt \hrule height 4pt \vskip4pt}%
   \vglue\titlepagetopglue
   %
   % Now you can put text using @subtitle.
   \def\subtitle{\parsearg\subtitlezzz}%
   \def\subtitlezzz##1{{\subtitlefont \rightline{##1}}}%
   %
   % @author should come last, but may come many times.
   \def\author{\parsearg\authorzzz}%
   \def\authorzzz##1{\ifseenauthor\else\vskip 0pt plus 1filll\seenauthortrue\fi
      {\authorfont \leftline{##1}}}%
   %  
   % Most title ``pages'' are actually two pages long, with space
   % at the top of the second.  We don't want the ragged left on the second.
   \let\oldpage = \page
%   \def\page{\vskip4pt \hrule height 2pt \vskip\titlepagebottomglue
%      \oldpage \endgroup\hrule height0pt\relax}%
   \def\page{\oldpage \hbox{}}}
}

\def\Etitlepage{\endgroup\page\HEADINGSon}

%%% Set up page headings and footings.

\let\thispage=\folio

\newtoks \evenheadline    % Token sequence for heading line of even pages
\newtoks \oddheadline     % Token sequence for heading line of odd pages
\newtoks \evenfootline    % Token sequence for footing line of even pages
\newtoks \oddfootline     % Token sequence for footing line of odd pages

% Now make Tex use those variables
\headline={{\textfonts\rm \ifodd\pageno \the\oddheadline \else \the\evenheadline \fi}}
\footline={{\textfonts\rm \ifodd\pageno \the\oddfootline \else \the\evenfootline \fi}}

% Commands to set those variables.
% For example, this is what  @headings on  does
% @evenheading @thistitle|@thispage|@thischapter
% @oddheading @thischapter|@thispage|@thistitle
% @evenfooting @thisfile||
% @oddfooting ||@thisfile

\def\evenheading{\parsearg\evenheadingxxx}
\def\oddheading{\parsearg\oddheadingxxx}
\def\everyheading{\parsearg\everyheadingxxx}

\def\evenfooting{\parsearg\evenfootingxxx}
\def\oddfooting{\parsearg\oddfootingxxx}
\def\everyfooting{\parsearg\everyfootingxxx}

{\catcode`\@=0 %

\gdef\evenheadingxxx #1{\evenheadingyyy #1@|@|@|@|\finish}
\gdef\evenheadingyyy #1@|#2@|#3@|#4\finish{%
\global\evenheadline={\rlap{\centerline{#2}}\line{#1\hfil#3}}}

\gdef\oddheadingxxx #1{\oddheadingyyy #1@|@|@|@|\finish}
\gdef\oddheadingyyy #1@|#2@|#3@|#4\finish{%
\global\oddheadline={\rlap{\centerline{#2}}\line{#1\hfil#3}}}

\gdef\everyheadingxxx #1{\everyheadingyyy #1@|@|@|@|\finish}
\gdef\everyheadingyyy #1@|#2@|#3@|#4\finish{%
\global\evenheadline={\rlap{\centerline{#2}}\line{#1\hfil#3}}
\global\oddheadline={\rlap{\centerline{#2}}\line{#1\hfil#3}}}

\gdef\evenfootingxxx #1{\evenfootingyyy #1@|@|@|@|\finish}
\gdef\evenfootingyyy #1@|#2@|#3@|#4\finish{%
\global\evenfootline={\rlap{\centerline{#2}}\line{#1\hfil#3}}}

\gdef\oddfootingxxx #1{\oddfootingyyy #1@|@|@|@|\finish}
\gdef\oddfootingyyy #1@|#2@|#3@|#4\finish{%
\global\oddfootline={\rlap{\centerline{#2}}\line{#1\hfil#3}}}

\gdef\everyfootingxxx #1{\everyfootingyyy #1@|@|@|@|\finish}
\gdef\everyfootingyyy #1@|#2@|#3@|#4\finish{%
\global\evenfootline={\rlap{\centerline{#2}}\line{#1\hfil#3}}
\global\oddfootline={\rlap{\centerline{#2}}\line{#1\hfil#3}}}
%
}% unbind the catcode of @.

% @headings double	turns headings on for double-sided printing.
% @headings single	turns headings on for single-sided printing.
% @headings off		turns them off.
% @headings on		same as @headings double, retained for compatibility.
% By default, they are off.

\def\headings #1 {\csname HEADINGS#1\endcsname}

\def\HEADINGSoff{
\global\evenheadline={\hfil} \global\evenfootline={\hfil}
\global\oddheadline={\hfil} \global\oddfootline={\hfil}}
\HEADINGSoff
% When we turn headings on, set the page number to 1.
% For double-sided printing, put current file name in lower left corner,
% chapter name on inside top of right hand pages, document
% title on inside top of left hand pages, and page numbers on outside top
% edge of all pages.
\def\HEADINGSdouble{
%\pagealignmacro
\global\pageno=1
\global\evenfootline={\hfil}
\global\oddfootline={\hfil}
\global\evenheadline={\line{\folio\hfil\thistitle}}
\global\oddheadline={\line{\thischapter\hfil\folio}}
}
% For single-sided printing, chapter title goes across top left of page,
% page number on top right.
\def\HEADINGSsingle{
%\pagealignmacro
\global\pageno=1
\global\evenfootline={\hfil}
\global\oddfootline={\hfil}
\global\evenheadline={\line{\thischapter\hfil\folio}}
\global\oddheadline={\line{\thischapter\hfil\folio}}
}
\def\HEADINGSon{\HEADINGSdouble}

% Subroutines used in generating headings
% Produces Day Month Year style of output.
\def\today{\number\day\space
\ifcase\month\or
January\or February\or March\or April\or May\or June\or
July\or August\or September\or October\or November\or December\fi
\space\number\year}

% Use this if you want the Month Day, Year style of output.
%\def\today{\ifcase\month\or
%January\or February\or March\or April\or May\or June\or
%July\or August\or September\or October\or November\or December\fi
%\space\number\day, \number\year}

% @settitle line...  specifies the title of the document, for headings
% It generates no output of its own

\def\thistitle{No Title}
\def\settitle{\parsearg\settitlezzz}
\def\settitlezzz #1{\gdef\thistitle{#1}}

\message{tables,}

% Tables -- @table, @ftable, @item(x), @kitem(x), @xitem(x).

% default indentation of table text
\newdimen\tableindent \tableindent=.8in
% default indentation of @itemize and @enumerate text
\newdimen\itemindent  \itemindent=.3in
% margin between end of table item and start of table text.
\newdimen\itemmargin  \itemmargin=.1in

% used internally for \itemindent minus \itemmargin
\newdimen\itemmax

% Note @table and @ftable define @item, @itemx, etc., with these defs.
% They also define \itemindex
% to index the item name in whatever manner is desired (perhaps none).

\def\internalBitem{\smallbreak \parsearg\itemzzz}
\def\internalBitemx{\par \parsearg\itemzzz}

\def\internalBxitem "#1"{\def\xitemsubtopix{#1} \smallbreak \parsearg\xitemzzz}
\def\internalBxitemx "#1"{\def\xitemsubtopix{#1} \par \parsearg\xitemzzz}

\def\internalBkitem{\smallbreak \parsearg\kitemzzz}
\def\internalBkitemx{\par \parsearg\kitemzzz}

\def\kitemzzz #1{\dosubind {kw}{\code{#1}}{for {\bf \lastfunction}}\itemzzz {#1}}

\def\xitemzzz #1{\dosubind {kw}{\code{#1}}{for {\bf \xitemsubtopic}}\itemzzz {#1}}

\def\itemzzz #1{\begingroup %
\advance \hsize by -\rightskip %
\advance \hsize by -\leftskip %
\setbox0=\hbox{\itemfont{#1}}%
\itemindex{#1}%
\parskip=0in %
\noindent %
\ifdim \wd0>\itemmax %
\vadjust{\penalty 10000}%
\hbox to \hsize{\hskip -\tableindent\box0\hss}\ %
\else %
\hbox to 0pt{\hskip -\tableindent\box0\hss}%
\fi %
\endgroup %
}

\def\item{\errmessage{@item while not in a table}}
\def\itemx{\errmessage{@itemx while not in a table}}
\def\kitem{\errmessage{@kitem while not in a table}}
\def\kitemx{\errmessage{@kitemx while not in a table}}
\def\xitem{\errmessage{@xitem while not in a table}}
\def\xitemx{\errmessage{@xitemx while not in a table}}

%% Contains a kludge to get @end[description] to work
\def\description{\tablez{\dontindex}{1}{}{}{}{}}

\def\table{\begingroup\inENV\obeylines\obeyspaces\tablex}
{\obeylines\obeyspaces%
\gdef\tablex #1^^M{%
\tabley\dontindex#1        \endtabley}}

\def\ftable{\begingroup\inENV\obeylines\obeyspaces\ftablex}
{\obeylines\obeyspaces%
\gdef\ftablex #1^^M{%
\tabley\fnitemindex#1        \endtabley}}

\def\dontindex #1{}
\def\fnitemindex #1{\doind {fn}{\code{#1}}}%

{\obeyspaces %
\gdef\tabley#1#2 #3 #4 #5 #6 #7\endtabley{\endgroup%
\tablez{#1}{#2}{#3}{#4}{#5}{#6}}}

\def\tablez #1#2#3#4#5#6{%
\aboveenvbreak %
\begingroup %
\def\Edescription{\Etable}% Neccessary kludge.
\let\itemindex=#1%
\ifnum 0#3>0 \advance \leftskip by #3\mil \fi %
\ifnum 0#4>0 \tableindent=#4\mil \fi %
\ifnum 0#5>0 \advance \rightskip by #5\mil \fi %
\def\itemfont{#2}%
\itemmax=\tableindent %
\advance \itemmax by -\itemmargin %
\advance \leftskip by \tableindent %
\parindent = 0pt
\parskip = \smallskipamount
\ifdim \parskip=0pt \parskip=2pt \fi%
\def\Etable{\endgraf\endgroup\afterenvbreak}%
\let\item = \internalBitem %
\let\itemx = \internalBitemx %
\let\kitem = \internalBkitem %
\let\kitemx = \internalBkitemx %
\let\xitem = \internalBxitem %
\let\xitemx = \internalBxitemx %
}

% This is the counter used by @enumerate, which is really @itemize

\newcount \itemno

\def\itemize{\parsearg\itemizezzz}

\def\itemizezzz #1{\itemizey {#1}{\Eitemize}}

\def\itemizey #1#2{%
\aboveenvbreak %
\begingroup %
\itemno = 0 %
\itemmax=\itemindent %
\advance \itemmax by -\itemmargin %
\advance \leftskip by \itemindent %
\parindent = 0pt
\parskip = \smallskipamount
\ifdim \parskip=0pt \parskip=2pt \fi%
\def#2{\endgraf\endgroup\afterenvbreak}%
\def\itemcontents{#1}%
\let\item=\itemizeitem}

\def\bullet{$\ptexbullet$}
\def\minus{$-$}

\def\enumerate{\itemizey{\the\itemno.}\Eenumerate\flushcr}

% Definition of @item while inside @itemize.

\def\itemizeitem{%
\advance\itemno by 1
{\let\par=\endgraf \smallbreak}%
\ifhmode \errmessage{\in hmode at itemizeitem}\fi
{\parskip=0in \hskip 0pt
\hbox to 0pt{\hss \itemcontents\hskip \itemmargin}%
\vadjust{\penalty 300}}%
\flushcr}

\message{indexing,}
% Index generation facilities

% Define \newwrite to be identical to plain tex's \newwrite
% except not \outer, so it can be used within \newindex.
{\catcode`\@=11
\gdef\newwrite{\alloc@7\write\chardef\sixt@@n}}

% \newindex {foo} defines an index named foo.
% It automatically defines \fooindex such that
% \fooindex ...rest of line... puts an entry in the index foo.
% It also defines \fooindfile to be the number of the output channel for
% the file that	accumulates this index.  The file's extension is foo.
% The name of an index should be no more than 2 characters long
% for the sake of vms.

\def\newindex #1{
\expandafter\newwrite \csname#1indfile\endcsname% Define number for output file
\openout \csname#1indfile\endcsname \jobname.#1	% Open the file
\expandafter\xdef\csname#1index\endcsname{%	% Define \xxxindex
\noexpand\doindex {#1}}
}

% @defindex foo  ==  \newindex{foo}

\def\defindex{\parsearg\newindex}

% Define @defcodeindex, like @defindex except put all entries in @code.

\def\newcodeindex #1{
\expandafter\newwrite \csname#1indfile\endcsname% Define number for output file
\openout \csname#1indfile\endcsname \jobname.#1	% Open the file
\expandafter\xdef\csname#1index\endcsname{%	% Define \xxxindex
\noexpand\docodeindex {#1}}
}

\def\defcodeindex{\parsearg\newcodeindex}

% @synindex foo bar    makes index foo feed into index bar.
% Do this instead of @defindex foo if you don't want it as a separate index.
\def\synindex #1 #2 {%
\expandafter\xdef\csname#1index\endcsname{%	% Define \xxxindex
\noexpand\doindex {#2}}%
}

% @syncodeindex foo bar   similar, but put all entries made for index foo
% inside @code.
\def\syncodeindex #1 #2 {%
\expandafter\xdef\csname#1index\endcsname{%	% Define \xxxindex
\noexpand\docodeindex {#2}}%
}

% Define \doindex, the driver for all \fooindex macros.
% Argument #1 is generated by the calling \fooindex macro,
%  and it is "foo", the name of the index.

% \doindex just uses \parsearg; it calls \doind for the actual work.
% This is because \doind is more useful to call from other macros.

% There is also \dosubind {index}{topic}{subtopic}
% which makes an entry in a two-level index such as the operation index.

\def\doindex#1{\edef\indexname{#1}\parsearg\singleindexer}
\def\singleindexer #1{\doind{\indexname}{#1}}

% like the previous two, but they put @code around the argument.
\def\docodeindex#1{\edef\indexname{#1}\parsearg\singlecodeindexer}
\def\singlecodeindexer #1{\doind{\indexname}{\code{#1}}}

\def\indexdummies{%
\def\bf{\realbackslash bf }%
\def\rm{\realbackslash rm }%
\def\sl{\realbackslash sl }%
\def\dots{\realbackslash dots }%
\def\copyright{\realbackslash copyright }%
}

% \indexnofonts no-ops all font-change commands.
% This is used when outputting the strings to sort the index by.
\def\indexdummyfont#1{#1}
\def\indexnofonts{%
\let\code=\indexdummyfont
\let\samp=\indexdummyfont
\let\kbd=\indexdummyfont
\let\key=\indexdummyfont
\let\var=\indexdummyfont
}

% To define \realbackslash, we must make \ not be an escape.
% We must first make another character (@) an escape
% so we do not become unable to do a definition.

{\catcode`\@=0 \catcode`\\=\other
@gdef@realbackslash{\}}

\let\indexbackslash=0  %overridden during \printindex.

\def\doind #1#2{%
{\indexdummies % Must do this here, since \bf, etc expand at this stage
\count10=\lastpenalty %
\escapechar=`\\%
{\let\folio=0% Expand all macros now EXCEPT \folio
\def\rawbackslashxx{\indexbackslash}% \indexbackslash isn't defined now
% so it will be output as is; and it will print as backslash in the indx.
%
% Now process the index-string once, with all font commands turned off,
% to get the string to sort the index by.
{\indexnofonts
\xdef\temp1{#2}%
}%
% Now produce the complete index entry.  We process the index-string again,
% this time with font commands expanded, to get what to print in the index.
\edef\temp{%
\write \csname#1indfile\endcsname{%
\realbackslash entry {\temp1}{\folio}{#2}}}%
\temp }%
\penalty\count10}}

\def\dosubind #1#2#3{%
{\indexdummies % Must do this here, since \bf, etc expand at this stage
\count10=\lastpenalty %
\escapechar=`\\%
{\let\folio=0%
\def\rawbackslashxx{\indexbackslash}%
%
% Now process the index-string once, with all font commands turned off,
% to get the string to sort the index by.
{\indexnofonts
\xdef\temp1{#2 #3}%
}%
% Now produce the complete index entry.  We process the index-string again,
% this time with font commands expanded, to get what to print in the index.
\edef\temp{%
\write \csname#1indfile\endcsname{%
\realbackslash entry {\temp1}{\folio}{#2}{#3}}}%
\temp }%
\penalty\count10}}

% The index entry written in the file actually looks like
%  \entry {sortstring}{page}{topic}
% or
%  \entry {sortstring}{page}{topic}{subtopic}
% The texindex program reads in these files and writes files
% containing these kinds of lines:
%  \initial {c}
%     before the first topic whose initial is c
%  \entry {topic}{pagelist}
%     for a topic that is used without subtopics
%  \primary {topic}
%     for the beginning of a topic that is used with subtopics
%  \secondary {subtopic}{pagelist}
%     for each subtopic.

% Define the user-accessible indexing commands 
% @findex, @vindex, @kindex, @cindex.

\def\findex {\fnindex}
\def\kindex {\kyindex}
\def\cindex {\cpindex}
\def\vindex {\vrindex}
\def\tindex {\tpindex}
\def\pindex {\pgindex}

\def\cindexsub {\begingroup\obeylines\cindexsub}
{\obeylines %
\gdef\cindexsub "#1" #2^^M{\endgroup %
\dosubind{cp}{#2}{#1}}}

% Define the macros used in formatting output of the sorted index material.

% This is what you call to cause a particular index to get printed.
% Write
% @unnumbered Function Index
% @printindex fn

\def\printindex{\parsearg\doprintindex}

\def\doprintindex#1{\tex %
\catcode`\%=\other\catcode`\&=\other\catcode`\#=\other
\catcode`\$=\other\catcode`\_=\other
\catcode`\~=\other
\def\indexbackslash{\rawbackslashxx}
\indexfonts\rm \tolerance=9500 \advance\baselineskip -1pt
\begindoublecolumns
\openin 1 \jobname.#1s
\ifeof 1 \else \closein 1 \input \jobname.#1s
\fi
\enddoublecolumns
\Etex}

% These macros are used by the sorted index file itself.
% Change them to control the appearance of the index.

% Same as \bigskipamount except no shrink.
% \balancecolumns gets confused if there is any shrink.
\newskip\initialskipamount \initialskipamount 12pt plus4pt

\outer\def\initial #1{%
{\let\tentt=\sectt \let\sf=\sectt
\ifdim\lastskip<\initialskipamount
\removelastskip \penalty-200 \vskip \initialskipamount\fi
\line{\secbf#1\hfill}\kern 2pt\penalty3000}}

\outer\def\entry #1#2{
{\parfillskip=0in \parskip=0in \parindent=0in
\hangindent=1in \hangafter=1%
\noindent\hbox{#1}\dotfill #2\par
}}

\def\primary #1{\line{#1\hfil}}

\newskip\secondaryindent \secondaryindent=0.5cm

\def\secondary #1#2{
{\parfillskip=0in \parskip=0in
\hangindent =1in \hangafter=1
\noindent\hskip\secondaryindent\hbox{#1}\dotfill #2\par
}}

%% Define two-column mode, which is used in indexes.
%% Adapted from the TeXBook, page 416
\catcode `\@=11

\newbox\partialpage

\newdimen\doublecolumnhsize  \doublecolumnhsize = 3.11in
\newdimen\doublecolumnvsize  \doublecolumnvsize = 19.1in

\def\begindoublecolumns{\begingroup
  \output={\global\setbox\partialpage=\vbox{\unvbox255\kern -\topskip \kern \baselineskip}}\eject
  \output={\doublecolumnout} \hsize=\doublecolumnhsize \vsize=\doublecolumnvsize}
\def\enddoublecolumns{\output={\balancecolumns}\eject
  \endgroup \pagegoal=\vsize}

\def\doublecolumnout{\splittopskip=\topskip \splitmaxdepth=\maxdepth
  \dimen@=\pageheight \advance\dimen@ by-\ht\partialpage
  \setbox0=\vsplit255 to\dimen@ \setbox2=\vsplit255 to\dimen@
  \onepageout\pagesofar \unvbox255 \penalty\outputpenalty}
\def\pagesofar{\unvbox\partialpage %
  \hsize=\doublecolumnhsize % have to restore this since output routine
%	      changes it to set cropmarks (P. A. MacKay, 12 Nov. 1986)
  \wd0=\hsize \wd2=\hsize \hbox to\pagewidth{\box0\hfil\box2}}
\def\balancecolumns{\setbox0=\vbox{\unvbox255} \dimen@=\ht0
  \advance\dimen@ by\topskip \advance\dimen@ by-\baselineskip
  \divide\dimen@ by2 \splittopskip=\topskip
  {\vbadness=10000 \loop \global\setbox3=\copy0
    \global\setbox1=\vsplit3 to\dimen@
    \ifdim\ht3>\dimen@ \global\advance\dimen@ by1pt \repeat}
  \setbox0=\vbox to\dimen@{\unvbox1}  \setbox2=\vbox to\dimen@{\unvbox3}
  \pagesofar}

\catcode `\@=\other
\message{sectioning,}
% Define chapters, sections, etc.

\newcount \chapno
\newcount \secno
\newcount \subsecno
\newcount \subsubsecno

% This counter is funny since it counts through charcodes of letters A, B, ...
\newcount \appendixno  \appendixno = `\@
\def\appendixletter{\char\the\appendixno}

\newwrite \contentsfile
% This is called from \setfilename.
\def\opencontents{\openout \contentsfile = \jobname.toc}

% Each @chapter defines this as the name of the chapter.
% page headings and footings can use it.  @section does likewise

\def\thischapter{} \def\thissection{}
\def\seccheck#1{\if \pageno<0 %
\errmessage{@#1 not allowed after generating table of contents}\fi
%
}

\outer\def\chapter{\parsearg\chapterzzz}
\def\chapterzzz #1{\seccheck{chapter}%
\secno=0 \subsecno=0 \subsubsecno=0 \global\advance \chapno by 1 \message{Chapter \the\chapno}%
\chapmacro {#1}{\the\chapno}%
\gdef\thissection{#1}\gdef\thischapter{#1}%
\let\rawbackslash=\relax%
\let\frenchspacing=\relax%
\edef\temp{{\realbackslash chapentry {#1}{\the\chapno}{\noexpand\folio}}}%
\escapechar=`\\%
\write \contentsfile \temp  %
\donoderef %
}

\outer\def\appendix{\parsearg\appendixzzz}
\def\appendixzzz #1{\seccheck{appendix}%
\secno=0 \subsecno=0 \subsubsecno=0 \global\advance \appendixno by 1 \message{Appendix \appendixletter}%
\chapmacro {#1}{Appendix \appendixletter}%
\gdef\thischapter{#1}\gdef\thissection{#1}%
\let\rawbackslash=\relax%
\let\frenchspacing=\relax%
\edef\temp{{\realbackslash chapentry {#1}{Appendix \appendixletter}{\noexpand\folio}}}%
\escapechar=`\\%
\write \contentsfile \temp  %
\unnumbnoderef %
}

\outer\def\unnumbered{\parsearg\unnumberedzzz}
\def\unnumberedzzz #1{\seccheck{unnumbered}%
\secno=0 \subsecno=0 \subsubsecno=0 \message{(#1)}
\unnumbchapmacro {#1}%
\gdef\thischapter{#1}\gdef\thissection{#1}%
\let\rawbackslash=\relax%
\let\frenchspacing=\relax%
\edef\temp{{\realbackslash unnumbchapentry {#1}{\noexpand\folio}}}%
\escapechar=`\\%
\write \contentsfile \temp  %
\unnumbnoderef %
}

\outer\def\section{\parsearg\sectionzzz}
\def\sectionzzz #1{\seccheck{section}%
\subsecno=0 \subsubsecno=0 \global\advance \secno by 1 %
\gdef\thissection{#1}\secheading {#1}{\the\chapno}{\the\secno}%
\let\rawbackslash=\relax%
\let\frenchspacing=\relax%
\edef\temp{{\realbackslash secentry %
{#1}{\the\chapno}{\the\secno}{\noexpand\folio}}}%
\escapechar=`\\%
\write \contentsfile \temp %
\donoderef %
\penalty 10000 %
}

\outer\def\appendixsection{\parsearg\appendixsectionzzz}
\outer\def\appendixsec{\parsearg\appendixsectionzzz}
\def\appendixsectionzzz #1{\seccheck{appendixsection}%
\subsecno=0 \subsubsecno=0 \global\advance \secno by 1 %
\gdef\thissection{#1}\secheading {#1}{\appendixletter}{\the\secno}%
\let\rawbackslash=\relax%
\let\frenchspacing=\relax%
\edef\temp{{\realbackslash secentry %
{#1}{\appendixletter}{\the\secno}{\noexpand\folio}}}%
\escapechar=`\\%
\write \contentsfile \temp %
\unnumbnoderef %
\penalty 10000 %
}

\outer\def\unnumberedsec{\parsearg\unnumberedseczzz}
\def\unnumberedseczzz #1{\seccheck{unnumberedsec}%
\plainsecheading {#1}\gdef\thissection{#1}%
\let\rawbackslash=\relax%
\let\frenchspacing=\relax%
\edef\temp{{\realbackslash unnumbsecentry{#1}{\noexpand\folio}}}%
\escapechar=`\\%
\write \contentsfile \temp %
\unnumbnoderef %
\penalty 10000 %
}

\outer\def\subsection{\parsearg\subsectionzzz}
\def\subsectionzzz #1{\seccheck{subsection}%
\gdef\thissection{#1}\subsubsecno=0 \global\advance \subsecno by 1 %
\subsecheading {#1}{\the\chapno}{\the\secno}{\the\subsecno}%
\let\rawbackslash=\relax%
\let\frenchspacing=\relax%
\edef\temp{{\realbackslash subsecentry %
{#1}{\the\chapno}{\the\secno}{\the\subsecno}{\noexpand\folio}}}%
\escapechar=`\\%
\write \contentsfile \temp %
\donoderef %
\penalty 10000 %
}

\outer\def\appendixsubsec{\parsearg\appendixsubseczzz}
\def\appendixsubseczzz #1{\seccheck{appendixsubsec}%
\gdef\thissection{#1}\subsubsecno=0 \global\advance \subsecno by 1 %
\subsecheading {#1}{\appendixletter}{\the\secno}{\the\subsecno}%
\let\rawbackslash=\relax%
\let\frenchspacing=\relax%
\edef\temp{{\realbackslash subsecentry %
{#1}{\appendixletter}{\the\secno}{\the\subsecno}{\noexpand\folio}}}%
\escapechar=`\\%
\write \contentsfile \temp %
\unnumbnoderef %
\penalty 10000 %
}

\outer\def\unnumberedsubsec{\parsearg\unnumberedsubseczzz}
\def\unnumberedsubseczzz #1{\seccheck{unnumberedsubsec}%
\plainsecheading {#1}\gdef\thissection{#1}%
\let\rawbackslash=\relax%
\let\frenchspacing=\relax%
\edef\temp{{\realbackslash unnumbsubsecentry{#1}{\noexpand\folio}}}%
\escapechar=`\\%
\write \contentsfile \temp %
\unnumbnoderef %
\penalty 10000 %
}

\outer\def\subsubsection{\parsearg\subsubsectionzzz}
\def\subsubsectionzzz #1{\seccheck{subsubsection}%
\gdef\thissection{#1}\global\advance \subsubsecno by 1 %
\subsubsecheading {#1}{\the\chapno}{\the\secno}{\the\subsecno}{\the\subsubsecno}%
\let\rawbackslash=\relax%
\let\frenchspacing=\relax%
\edef\temp{{\realbackslash subsubsecentry %
{#1}{\the\chapno}{\the\secno}{\the\subsecno}{\the\subsubsecno}{\noexpand\folio}}}%\
\escapechar=`\\%
\write \contentsfile \temp %
\donoderef %
\penalty 10000 %
}

\outer\def\appendixsubsubsec{\parsearg\appendixsubsubseczzz}
\def\appendixsubsubseczzz #1{\seccheck{appendixsubsubsec}%
\gdef\thissection{#1}\global\advance \subsubsecno by 1 %
\subsubsecheading {#1}{\appendixletter}{\the\secno}{\the\subsecno}{\the\subsubsecno}%
\let\rawbackslash=\relax%
\let\frenchspacing=\relax%
\edef\temp{{\realbackslash subsubsecentry{#1}%
{\appendixletter}{\the\secno}{\the\subsecno}{\the\subsubsecno}{\noexpand\folio}}}%\
\escapechar=`\\%
\write \contentsfile \temp %
\unnumbnoderef %
\penalty 10000 %
}

\outer\def\unnumberedsubsubsec{\parsearg\unnumberedsubsubseczzz}
\def\unnumberedsubsubseczzz #1{\seccheck{unnumberedsubsubsec}%
\plainsecheading {#1}\gdef\thissection{#1}%
\let\rawbackslash=\relax%
\let\frenchspacing=\relax%
\edef\temp{{\realbackslash unnumbsubsubsecentry{#1}{\noexpand\folio}}}%
\escapechar=`\\%
\write \contentsfile \temp %
\unnumbnoderef %
\penalty 10000 %
}

% These are variants which are not "outer", so they can appear in @ifinfo.
\def\infounnumbered{\parsearg\unnumberedzzz}
\def\infounnumberedsec{\parsearg\unnumberedseczzz}
\def\infounnumberedsubsec{\parsearg\unnumberedsubseczzz}
\def\infounnumberedsubsubsec{\parsearg\unnumberedsubsubseczzz}

\def\infoappendix{\parsearg\appendixzzz}
\def\infoappendixsec{\parsearg\appendixseczzz}
\def\infoappendixsubsec{\parsearg\appendixsubseczzz}
\def\infoappendixsubsubsec{\parsearg\appendixsubsubseczzz}

\def\infochapter{\parsearg\chapterzzz}
\def\infosection{\parsearg\sectionzzz}
\def\infosubsection{\parsearg\subsectionzzz}
\def\infosubsubsection{\parsearg\subsubsectionzzz}

% Define @majorheading, @heading and @subheading

\def\majorheading #1{%
{\advance\chapheadingskip by 10pt \chapbreak }%
{\chapfonts \line{\rm #1\hfill}}\bigskip \par\penalty 200}

\def\chapheading #1{\chapbreak %
{\chapfonts \line{\rm #1\hfill}}\bigskip \par\penalty 200}

\def\heading{\parsearg\secheadingi}

% These macros generate a chapter, section, etc. heading only
% (including whitespace, linebreaking, etc. around it),
% given all the information in convenient, parsed form.

%%% Args are the skip and penalty (usually negative)
\def\dobreak#1#2{\par\ifdim\lastskip<#1\removelastskip\penalty#2\vskip#1\fi}

\def\setchapterstyle #1 {\csname CHAPF#1\endcsname}

%%% Define plain chapter starts, and page on/off switching for it
% Parameter controlling skip before chapter headings (if needed)

\newskip \chapheadingskip \chapheadingskip = 30pt plus 8pt minus 4pt

\def\chapbreak{\dobreak \chapheadingskip {-4000}}
\def\chappager{\par\vfill\supereject}
\def\chapoddpage{\chappager \ifodd\pageno \else \hbox to 0pt{} \chappager\fi}

\def\setchapternewpage #1 {\csname CHAPPAG#1\endcsname}

\def\CHAPPAGoff{
\global\let\pchapsepmacro=\chapbreak
\global\let\pagealignmacro=\chappager}

\def\CHAPPAGon{
\global\let\pchapsepmacro=\chappager
\global\let\pagealignmacro=\chappager
\global\def\HEADINGSon{\HEADINGSsingle}}

\def\CHAPPAGodd{
\global\let\pchapsepmacro=\chapoddpage
\global\let\pagealignmacro=\chapoddpage
\global\def\HEADINGSon{\HEADINGSdouble}}

\CHAPPAGon

\def\CHAPFplain{
\global\let\chapmacro=\chfplain
\global\let\unnumbchapmacro=\unnchfplain}

\def\chfplain #1#2{%
\pchapsepmacro %
{\chapfonts \line{\rm #2.\enspace #1\hfill}}\bigskip \par\penalty 5000 %
}

\def\unnchfplain #1{%
\pchapsepmacro %
{\chapfonts \line{\rm #1\hfill}}\bigskip \par\penalty 10000 %
}
\CHAPFplain % The default

\def\unnchfopen #1{%
\chapoddpage {\chapfonts \line{\rm #1\hfill}}\bigskip \par\penalty 10000 %
}

\def\chfopen #1#2{\chapoddpage {\chapfonts
\vbox to 3in{\vfil \hbox to\hsize{\hfil #2} \hbox to\hsize{\hfil #1} \vfil}}%
\par\penalty 5000 %
}

\def\CHAPFopen{
\global\let\chapmacro=\chfopen
\global\let\unnumbchapmacro=\unnchfopen}

% Parameter controlling skip before section headings.

\newskip \subsecheadingskip  \subsecheadingskip = 17pt plus 8pt minus 4pt
\def\subsecheadingbreak{\dobreak \subsecheadingskip {-500}}

\newskip \secheadingskip  \secheadingskip = 21pt plus 8pt minus 4pt
\def\secheadingbreak{\dobreak \secheadingskip {-1000}}


% Section fonts are the base font at magstep2, which produces
% a size a bit more than 14 points in the default situation.	

\def\secheading #1#2#3{\secheadingi {#2.#3\enspace #1}}
\def\plainsecheading #1{\secheadingi {#1}}
\def\secheadingi #1{{\advance \secheadingskip by \parskip %
\secheadingbreak}%
{\secfonts \line{\rm #1\hfill}}%
\ifdim \parskip<10pt \kern 10pt\kern -\parskip\fi \penalty 10000 }


% Subsection fonts are the base font at magstep1, 
% which produces a size of 12 points.

\def\subsecheading #1#2#3#4{{\advance \subsecheadingskip by \parskip %
\subsecheadingbreak}%
{\subsecfonts \line{\rm#2.#3.#4\enspace #1\hfill}}%
\ifdim \parskip<10pt \kern 10pt\kern -\parskip\fi \penalty 10000 }

\def\subsubsecfonts{\subsecfonts} % Maybe this should change:
				  % Perhaps make sssec fonts scaled
				  % magstep half
\def\subsubsecheading #1#2#3#4#5{{\advance \subsecheadingskip by \parskip %
\subsecheadingbreak}%
{\subsubsecfonts \line{\rm#2.#3.#4.#5\enspace #1\hfill}}%
\ifdim \parskip<10pt \kern 10pt\kern -\parskip\fi \penalty 10000}


\message{toc printing,}

% Finish up the main text and prepare to read what we've written
% to \contentsfile.

\def\startcontents#1{%
   \ifnum \pageno>0
      \pagealignmacro
      \immediate\closeout \contentsfile
      \pageno = -1		% Request roman numbered pages.
   \fi
   \unnumbchapmacro{#1}\def\thischapter{#1}%
   \begingroup   		% Set up to handle contents files properly.
      \catcode`\\=0  \catcode`\{=1  \catcode`\}=2  \catcode`\@=11
      \raggedbottom             % Worry more about breakpoints than the bottom.
      \advance\hsize by -1in    % Don't use the full line length.
}

  
% Normal (long) toc.
\outer\def\contents{%
   \startcontents{Table of Contents}%
      \input \jobname.toc
   \endgroup
   \vfill \eject
}

% And just the chapters.
\outer\def\summarycontents{%
   \startcontents{Short Contents}%
      %
      \let\chapentry = \shortchapentry
      \let\unnumbchapentry = \shortunnumberedentry
      % We want a true roman here for the page numbers.
      \secfonts \let\rm = \truesecrm \rm
      \advance\baselineskip by 1pt % Open it up a little.
      \def\secentry ##1##2##3##4{}
      \def\unnumbsecentry ##1##2{}
      \def\subsecentry ##1##2##3##4##5{}
      \def\unnumbsubsecentry ##1##2{}
      \def\subsubsecentry ##1##2##3##4##5##6{}
      \def\unnumbsubsubsecentry ##1##2{}
      \input \jobname.toc
   \endgroup
   \vfill \eject
}
\let\shortcontents = \summarycontents

% These macros generate individual entries in the table of contents.
% The first argument is the chapter or section name.
% The last argument is the page number.
% The arguments in between are the chapter number, section number, ...

% Chapter-level things, for both the long and short contents.
\def\chapentry#1#2#3{\dochapentry{#2\labelspace#1}{#3}}
\def\shortchapentry#1#2#3{%
   \line{{#2\labelspace #1}\dotfill\doshortpageno{#3}}%
}

\def\unnumbchapentry#1#2{\dochapentry{#1}{#2}}
\def\shortunnumberedentry#1#2{%
   \line{#1\dotfill\doshortpageno{#2}}%
}

% Sections.
\def\secentry#1#2#3#4{\dosecentry{#2.#3\labelspace#1}{#4}}
\def\unnumbsecentry#1#2{\dosecentry{#1}{#2}}

% Subsections.
\def\subsecentry#1#2#3#4#5{\dosubsecentry{#2.#3.#4\labelspace#1}{#5}}
\def\unnumbsubsecentry#1#2{\dosubsecentry{#1}{#2}}

% And subsubsections.
\def\subsubsecentry#1#2#3#4#5#6{\dosubsubsecentry{#2.#3.#4.#5\labelspace#1}{#6}}
\def\unnumbsubsecentry#1#2{\dosubsubsecentry{#1}{#2}}


% This parameter controls the indentation of the various levels.
\newdimen\tocindent \tocindent = 3pc

% Now for the actual typesetting. In all these, #1 is the text and #2 is the 
% page number.
%
% If the toc has to be broken over pages, we would want to be at chapters 
% if at all possible; hence the \penalty.
\def\dochapentry#1#2{%
   \penalty-300 \vskip\baselineskip
   \line{\chapentryfonts #1\dotfill \dopageno{#2}}%
   \nobreak\vskip .25\baselineskip
}

\def\dosecentry#1#2{%
   \line{\secentryfonts \hskip\tocindent #1\dotfill \dopageno{#2}}%
}

\def\dosubsecentry#1#2{%
   \line{\subsecentryfonts \hskip2\tocindent #1\dotfill \dopageno{#2}}%
}

\def\dosubsubsecentry#1#2{%
   \line{\subsubsecentryfonts \hskip3\tocindent #1\dotfill \dopageno{#2}}%
}

% Space between chapter (or whatever) number and the title.
\def\labelspace{\hskip1em \relax}

\def\dopageno#1{{\rm #1}}
\def\doshortpageno#1{{\rm #1}}

\def\chapentryfonts{\secfonts \let\rm = \sf \sf}
\def\secentryfonts{\textfonts}
\let\subsecentryfonts = \textfonts
\let\subsubsecentryfonts = \textfonts


\message{environments,}

% @tex ... @end tex    escapes into raw Tex temporarily.
% One exception: @ is still an escape character, so that @end tex works.
% But \@ or @@ will get a plain tex @ character.

\def\tex{\begingroup
\catcode `\\=0 \catcode `\{=1 \catcode `\}=2
\catcode `\$=3 \catcode `\&=4 \catcode `\#=6
\catcode `\^=7 \catcode `\_=8 \catcode `\~=13 \let~=\tie
\catcode `\%=14
\catcode`\"=12
\catcode`\|=12
\catcode`\<=12
\catcode`\>=12
\escapechar=`\\
%
\let\{=\ptexlbrace
\let\}=\ptexrbrace
\let\.=\ptexdot
\let\*=\ptexstar
\def\@={@}%
\let\bullet=\ptexbullet
\let\b=\ptexb \let\c=\ptexc \let\i=\ptexi \let\t=\ptext \let\l=\ptexl
\let\L=\ptexL
%
\let\Etex=\endgroup}

% Define @lisp ... @endlisp.
% @lisp does a \begingroup so it can rebind things,
% including the definition of @endlisp (which normally is erroneous).

% Amount to narrow the margins by for @lisp.
\newskip\lispnarrowing \lispnarrowing=0.4in

% This is the definition that ^M gets inside @lisp
% phr: changed space to \null, to avoid overfull hbox problems.
{\obeyspaces%
\gdef\lisppar{\null\endgraf}}

% Cause \obeyspaces to make each Space cause a word-separation
% rather than the default which is that it acts punctuation.
% This is because space in tt font looks funny.
{\obeyspaces %
\gdef\sepspaces{\def {\ }}}

\newskip\aboveenvskipamount \aboveenvskipamount= 0pt
\def\aboveenvbreak{{\advance\aboveenvskipamount by \parskip
\endgraf \ifdim\lastskip<\aboveenvskipamount
\removelastskip \penalty-50 \vskip\aboveenvskipamount \fi}}

\def\afterenvbreak{\endgraf \ifdim\lastskip<\aboveenvskipamount
\removelastskip \penalty-50 \vskip\aboveenvskipamount \fi}

\def\lisp{\aboveenvbreak\begingroup\inENV %This group ends at the end of the @lisp body
\hfuzz=12truept % Don't be fussy
% Make spaces be word-separators rather than space tokens.
\sepspaces %
% Single space lines
\singlespace %
% The following causes blank lines not to be ignored
% by adding a space to the end of each line.
\let\par=\lisppar
\def\Elisp{\endgroup\afterenvbreak}%
\parskip=0pt
\advance \leftskip by \lispnarrowing
\parindent=0pt
\let\exdent=\internalexdent
\obeyspaces \obeylines \tt \rawbackslash
\def\next##1{}\next}


\let\example=\lisp
\def\Eexample{\Elisp}

\let\smallexample=\lisp
\def\Esmallexample{\Elisp}

% Macro for 9 pt. examples, necessary to print with 5" lines.
% From Pavel@xerox.  This is not really used unless the
% @smallbook command is given.

\def\smalllispx{\aboveenvbreak\begingroup\inENV
%			This group ends at the end of the @lisp body
\hfuzz=12truept % Don't be fussy
% Make spaces be word-separators rather than space tokens.
\sepspaces %
% Single space lines
\singlespace %
% The following causes blank lines not to be ignored
% by adding a space to the end of each line.
\let\par=\lisppar
\def\Esmalllisp{\endgroup\afterenvbreak}%
\parskip=0pt
\advance \leftskip by \lispnarrowing
\parindent=0pt
\let\exdent=\internalexdent
\obeyspaces \obeylines \ninett \rawbackslash
\def\next##1{}\next}

% This is @display; same as @lisp except use roman font.

\def\display{\begingroup\inENV %This group ends at the end of the @display body
\aboveenvbreak
% Make spaces be word-separators rather than space tokens.
\sepspaces %
% Single space lines
\singlespace %
% The following causes blank lines not to be ignored
% by adding a space to the end of each line.
\let\par=\lisppar
\def\Edisplay{\endgroup\afterenvbreak}%
\parskip=0pt
\advance \leftskip by \lispnarrowing
\parindent=0pt
\let\exdent=\internalexdent
\obeyspaces \obeylines
\def\next##1{}\next}

% This is @format; same as @lisp except use roman font and don't narrow margins

\def\format{\begingroup\inENV %This group ends at the end of the @format body
\aboveenvbreak
% Make spaces be word-separators rather than space tokens.
\sepspaces %
\singlespace %
% The following causes blank lines not to be ignored
% by adding a space to the end of each line.
\let\par=\lisppar
\def\Eformat{\endgroup\afterenvbreak}
\parskip=0pt \parindent=0pt
\obeyspaces \obeylines
\def\next##1{}\next}

% @flushleft and @flushright

\def\flushleft{\begingroup\inENV %This group ends at the end of the @format body
\aboveenvbreak
% Make spaces be word-separators rather than space tokens.
\sepspaces %
% The following causes blank lines not to be ignored
% by adding a space to the end of each line.
% This also causes @ to work when the directive name
% is terminated by end of line.
\let\par=\lisppar
\def\Eflushleft{\endgroup\afterenvbreak}%
\parskip=0pt \parindent=0pt
\obeyspaces \obeylines
\def\next##1{}\next}

\def\flushright{\begingroup\inENV %This group ends at the end of the @format body
\aboveenvbreak
% Make spaces be word-separators rather than space tokens.
\sepspaces %
% The following causes blank lines not to be ignored
% by adding a space to the end of each line.
% This also causes @ to work when the directive name
% is terminated by end of line.
\let\par=\lisppar
\def\Eflushright{\endgroup\afterenvbreak}%
\parskip=0pt \parindent=0pt
\advance \leftskip by 0pt plus 1fill
\obeyspaces \obeylines
\def\next##1{}\next}

% @quotation - narrow the margins.

\def\quotation{\begingroup\inENV %This group ends at the end of the @quotation body
{\parskip=0pt  % because we will skip by \parskip too, later
\aboveenvbreak}%
\singlespace
\parindent=0pt
\def\Equotation{\par\endgroup\afterenvbreak}%
\advance \rightskip by \lispnarrowing 
\advance \leftskip by \lispnarrowing}

\message{defuns,}
% Define formatter for defuns
% First, allow user to change definition object font (\df) internally
\def\setdeffont #1 {\csname DEF#1\endcsname}

\newskip\defbodyindent \defbodyindent=36pt
\newskip\defargsindent \defargsindent=50pt
\newskip\deftypemargin \deftypemargin=12pt
\newskip\deflastargmargin \deflastargmargin=18pt

\newcount\parencount
% define \functionparens, which makes ( and ) and & do special things.
% \functionparens affects the group it is contained in.
\def\activeparens{%
\catcode`\(=\active \catcode`\)=\active \catcode`\&=\active
\catcode`\[=\active \catcode`\]=\active}
{\activeparens % Now, smart parens don't turn on until &foo (see \amprm)
\gdef\functionparens{\boldbrax\let&=\amprm\parencount=0 }
\gdef\boldbrax{\let(=\opnr\let)=\clnr\let[=\lbrb\let]=\rbrb}

% Definitions of (, ) and & used in args for functions.
% This is the definition of ( outside of all parentheses.
\gdef\oprm#1 {{\rm\char`\(}#1 \bf \let(=\opnested %
\global\advance\parencount by 1 }
%
% This is the definition of ( when already inside a level of parens.
\gdef\opnested{\char`\(\global\advance\parencount by 1 }
%
\gdef\clrm{% Print a paren in roman if it is taking us back to depth of 0.
% also in that case restore the outer-level definition of (.
\ifnum \parencount=1 {\rm \char `\)}\sl \let(=\oprm \else \char `\) \fi
\global\advance \parencount by -1 }
% If we encounter &foo, then turn on ()-hacking afterwards
\gdef\amprm#1 {{\rm\&#1}\let(=\oprm \let)=\clrm\ }
%
\gdef\normalparens{\boldbrax\let&=\ampnr}
} % End of definition inside \activeparens
%% These parens (in \boldbrax) actually are a little bolder than the
%% contained text.  This is especially needed for [ and ]
\def\opnr{{\sf\char`\(}} \def\clnr{{\sf\char`\)}} \def\ampnr{\&}
\def\lbrb{{\tt\char`\[}} \def\rbrb{{\tt\char`\]}}

% First, defname, which formats the header line itself.
% #1 should be the function name.
% #2 should be the type of definition, such as "Function".

\def\defname #1#2{%
\leftskip = 0in  %
\noindent        %
\setbox0=\hbox{\hskip \deflastargmargin{\rm #2}\hskip \deftypemargin}%
\dimen0=\hsize \advance \dimen0 by -\wd0 % compute size for first line
\dimen1=\hsize \advance \dimen1 by -\defargsindent %size for continuations
\parshape 2 0in \dimen0 \defargsindent \dimen1     %
% Now output arg 2 ("Function" or some such)
% ending at \deftypemargin from the right margin,
% but stuck inside a box of width 0 so it does not interfere with linebreaking
\rlap{\rightline{{\rm #2}\hskip \deftypemargin}}%
\tolerance=10000 \hbadness=10000    % Make all lines underfull and no complaints
{\df #1}\enskip        % Generate function name
}

% Actually process the body of a definition
% #1 should be the terminating control sequence, such as \Edefun.
% #2 should be the "another name" control sequence, such as \defunx.
% #3 should be the control sequence that actually processes the header,
%    such as \defunheader.

\def\defparsebody #1#2#3{\begingroup\inENV% Environment for definitionbody
\medbreak %
% Define the end token that this defining construct specifies
% so that it will exit this group.
\def#1{\endgraf\endgroup\medbreak}%
\def#2{\begingroup\obeylines\activeparens\spacesplit#3}%
\parindent=0in \leftskip=\defbodyindent %
\begingroup\obeylines\activeparens\spacesplit#3}

\def\defmethparsebody #1#2#3#4 {\begingroup\inENV %
\medbreak %
% Define the end token that this defining construct specifies
% so that it will exit this group.
\def#1{\endgraf\endgroup\medbreak}%
\def#2##1 {\begingroup\obeylines\activeparens\spacesplit{#3{##1}}}%
\parindent=0in \leftskip=\defbodyindent %
\begingroup\obeylines\activeparens\spacesplit{#3{#4}}}

\def\defopparsebody #1#2#3#4#5 {\begingroup\inENV %
\medbreak %
% Define the end token that this defining construct specifies
% so that it will exit this group.
\def#1{\endgraf\endgroup\medbreak}%
\def#2##1 ##2 {\def#4{##1}%
\begingroup\obeylines\activeparens\spacesplit{#3{##2}}}%
\parindent=0in \leftskip=\defbodyindent %
\begingroup\obeylines\activeparens\spacesplit{#3{#5}}}

% Split up #2 at the first space token.
% call #1 with two arguments:
%  the first is all of #2 before the space token,
%  the second is all of #2 after that space token.
% If #2 contains no space token, all of it is passed as the first arg
% and the second is passed as empty.

{\obeylines
\gdef\spacesplit#1#2^^M{\endgroup\spacesplitfoo{#1}#2 \relax\spacesplitfoo}%
\long\gdef\spacesplitfoo#1#2 #3#4\spacesplitfoo{%
\ifx\relax #3%
#1{#2}{}\else #1{#2}{#3#4}\fi}}

% So much for the things common to all kinds of definitions.

% Define @defun.

% First, define the processing that is wanted for arguments of \defun
% Use this to expand the args and terminate the paragraph they make up

\def\defunargs #1{\functionparens \sl #1%
\ifnum\parencount=0 \else \errmessage{unbalanced parens in @def arguments}\fi%
\interlinepenalty=10000
\endgraf\vskip -\parskip \penalty 10000}

% Do complete processing of one @defun or @defunx line already parsed.

% @deffn Command forward-char nchars

\def\deffn{\defmethparsebody\Edeffn\deffnx\deffnheader}

\def\deffnheader #1#2#3{\doind {fn}{\code{#2}}%
\begingroup\defname {#2}{#1}\defunargs{#3}\endgroup}

% @defun == @deffn Function

\def\defun{\defparsebody\Edefun\defunx\defunheader}

\def\defunheader #1#2{\doind {fn}{\code{#1}}% Make entry in function index
\begingroup\defname {#1}{Function}%
\defunargs {#2}\endgroup %
}

% @defmac == @deffn Macro

\def\defmac{\defparsebody\Edefmac\defmacx\defmacheader}

\def\defmacheader #1#2{\doind {fn}{\code{#1}}% Make entry in function index
\begingroup\defname {#1}{Macro}%
\defunargs {#2}\endgroup %
}

% @defspec == @deffn Special Form

\def\defspec{\defparsebody\Edefspec\defspecx\defspecheader}

\def\defspecheader #1#2{\doind {fn}{\code{#1}}% Make entry in function index
\begingroup\defname {#1}{Special form}%
\defunargs {#2}\endgroup %
}

% This definition is run if you use @defunx
% anywhere other than immediately after a @defun or @defunx.

\def\deffnx #1 {\errmessage{@deffnx in invalid context}}
\def\defunx #1 {\errmessage{@defunx in invalid context}}
\def\defmacx #1 {\errmessage{@defmacx in invalid context}}
\def\defspecx #1 {\errmessage{@defspecx in invalid context}}

% @defmethod, and so on

% @defop {Funny Method} foo-class frobnicate argument

\def\defop #1 {\def\defoptype{#1}%
\defopparsebody\Edefop\defopx\defopheader\defoptype}

\def\defopheader #1#2#3{\dosubind {fn}{\code{#2}}{on #1}% Make entry in function index
\begingroup\defname {#2}{\defoptype{} on #1}%
\defunargs {#3}\endgroup %
}

% @defmethod == @defop Method

\def\defmethod{\defmethparsebody\Edefmethod\defmethodx\defmethodheader}

\def\defmethodheader #1#2#3{\dosubind {fn}{\code{#2}}{on #1}% entry in function index
\begingroup\defname {#2}{Operation on #1}%
\defunargs {#3}\endgroup %
}

% @defcv {Class Option} foo-class foo-flag

\def\defcv #1 {\def\defcvtype{#1}%
\defopparsebody\Edefcv\defcvx\defcvheader\defcvtype}

\def\defcvarheader #1#2#3{%
\dosubind {vr}{\code{#2}}{of #1}% Make entry in var index
\begingroup\defname {#2}{\defcvtype of #1}%
\defvarargs {#3}\endgroup %
}

% @defivar == @defcv {Instance Variable}

\def\defivar{\defmethparsebody\Edefivar\defivarx\defivarheader}

\def\defivarheader #1#2#3{%
\dosubind {vr}{\code{#2}}{of #1}% Make entry in var index
\begingroup\defname {#2}{Instance variable of #1}%
\defvarargs {#3}\endgroup %
}

% These definitions are run if you use @defmethodx, etc.,
% anywhere other than immediately after a @defmethod, etc.

\def\defopx #1 {\errmessage{@defopx in invalid context}}
\def\defmethodx #1 {\errmessage{@defmethodx in invalid context}}
\def\defcvx #1 {\errmessage{@defcvx in invalid context}}
\def\defivarx #1 {\errmessage{@defivarx in invalid context}}

% Now @defvar

% First, define the processing that is wanted for arguments of @defvar.
% This is actually simple: just print them in roman.
% This must expand the args and terminate the paragraph they make up
\def\defvarargs #1{\normalparens #1%
\interlinepenalty=10000
\endgraf\vskip -\parskip \penalty 10000}

% @defvr Counter foo-count

\def\defvr{\defmethparsebody\Edefvr\defvrx\defvrheader}

\def\defvrheader #1#2#3{\doind {vr}{\code{#2}}%
\begingroup\defname {#2}{#1}\defvarargs{#3}\endgroup}

% @defvar == @defvr Variable

\def\defvar{\defparsebody\Edefvar\defvarx\defvarheader}

\def\defvarheader #1#2{\doind {vr}{\code{#1}}% Make entry in var index
\begingroup\defname {#1}{Variable}%
\defvarargs {#2}\endgroup %
}

% @defopt == @defvr {User Option}

\def\defopt{\defparsebody\Edefopt\defoptx\defoptheader}

\def\defoptheader #1#2{\doind {vr}{\code{#1}}% Make entry in var index
\begingroup\defname {#1}{User Option}%
\defvarargs {#2}\endgroup %
}

% This definition is run if you use @defvarx
% anywhere other than immediately after a @defvar or @defvarx.

\def\defvrx #1 {\errmessage{@defvrx in invalid context}}
\def\defvarx #1 {\errmessage{@defvarx in invalid context}}
\def\defoptx #1 {\errmessage{@defoptx in invalid context}}

% Now define @deftp
% Args are printed in bold, a slight difference from @defvar.

\def\deftpargs #1{\bf \defvarargs{#1}}

% @deftp Class window height width ...

\def\deftp{\defmethparsebody\Edeftp\deftpx\deftpheader}

\def\deftpheader #1#2#3{\doind {tp}{\code{#2}}%
\begingroup\defname {#2}{#1}\deftpargs{#3}\endgroup}

% This definition is run if you use @deftpx, etc
% anywhere other than immediately after a @deftp, etc.

\def\deftpx #1 {\errmessage{@deftpx in invalid context}}

\message{cross reference,}
% Define cross-reference macros
\newwrite \auxfile

% \setref{foo} defines a cross-reference point named foo.

\def\setref#1{%
\dosetq{#1-pg}{Ypagenumber}%
\dosetq{#1-snt}{Ysectionnumberandtype}}

\def\unnumbsetref#1{%
\dosetq{#1-pg}{Ypagenumber}%
\dosetq{#1-snt}{Ynothing}}

% \xref and \pxref generate cross references to specified points.

\def\pxref #1{see \xrefX [#1,,,,,,,]}
\def\xref #1{See \xrefX [#1,,,,,,,]}
\def\xrefX [#1,#2,#3,#4,#5,#6]{%
\setbox1=\hbox{\i{\losespace#5{}}}%
\setbox0=\hbox{\losespace#3{}}%
\ifdim \wd0 =0pt \setbox0=\hbox{\losespace#1{}}\fi%
\ifdim \wd1 >0pt%
section \unhbox0{} in \unhbox1%
\else%
\refx{#1-snt} [\unhbox0], page\tie \refx{#1-pg}%
\fi }

% \dosetq is the interface for calls from other macros

\def\dosetq #1#2{{\let\folio=0%
\edef\next{\write\auxfile{\internalsetq {#1}{#2}}}%
\next}}

% \internalsetq {foo}{page} expands into CHARACTERS 'xrdef {foo}{...expansion of \Ypage...}
% When the aux file is read, ' is the escape character

\def\internalsetq #1#2{'xrdef {#1}{\csname #2\endcsname}}

% Things to be expanded by \internalsetq

\def\Ypagenumber{\folio}

\def\Ynothing{}

\def\Ysectionnumberandtype{%
\ifnum\secno=0 chapter\xreftie\the\chapno %
\else \ifnum \subsecno=0 section\xreftie\the\chapno.\the\secno %
\else \ifnum \subsubsecno=0 %
section\xreftie\the\chapno.\the\secno.\the\subsecno %
\else %
section\xreftie\the\chapno.\the\secno.\the\subsecno.\the\subsubsecno %
\fi \fi \fi }

\gdef\xreftie{'tie}

% Define @refx to reference a specific cross-reference string.

\def\refx#1{%
{%
\expandafter\ifx\csname X#1\endcsname\relax
% If not defined, say something at least.
\expandafter\gdef\csname X#1\endcsname {$<$undefined$>$}%
\message {WARNING: Cross-reference "#1" used but not yet defined}%
\message {}%
\fi %
\csname X#1\endcsname %It's defined, so just use it.
}}

% Read the last existing aux file, if any.  No error if none exists.

% This is the macro invoked by entries in the aux file.
\def\xrdef #1#2{
{\catcode`\'=\other\expandafter \gdef \csname X#1\endcsname {#2}}}

\def\readauxfile{%
\begingroup
\catcode `\^^@=\other
\catcode `\=\other
\catcode `\=\other
\catcode `\^^C=\other
\catcode `\^^D=\other
\catcode `\^^E=\other
\catcode `\^^F=\other
\catcode `\^^G=\other
\catcode `\^^H=\other
\catcode `\=\other
\catcode `\^^L=\other
\catcode `\=\other
\catcode `\=\other
\catcode `\=\other
\catcode `\=\other
\catcode `\=\other
\catcode `\=\other
\catcode `\=\other
\catcode `\=\other
\catcode `\=\other
\catcode `\=\other
\catcode `\=\other
\catcode `\=\other
\catcode `\=\other
\catcode `\^^[=\other
\catcode `\^^\=\other
\catcode `\^^]=\other
\catcode `\^^^=\other
\catcode `\^^_=\other
\catcode `\@=\other
\catcode `\^=\other
\catcode `\~=\other
\catcode `\[=\other
\catcode `\]=\other
\catcode`\"=\other
\catcode`\_=\other
\catcode`\|=\other
\catcode`\<=\other
\catcode`\>=\other
\catcode `\$=\other
\catcode `\#=\other
\catcode `\&=\other
% the aux file uses ' as the escape.
% Turn off \ as an escape so we do not lose on
% entries which were dumped with control sequences in their names.
% For example, 'xrdef {$\leq $-fun}{page ...} made by @defun ^^
% Reference to such entries still does not work the way one would wish,
% but at least they do not bomb out when the aux file is read in.
\catcode `\{=1 \catcode `\}=2
\catcode `\%=\other
\catcode `\'=0
\catcode `\\=\other
\openin 1 \jobname.aux
\ifeof 1 \else \closein 1 \input \jobname.aux
\fi
% Open the new aux file.  Tex will close it automatically at exit.
\openout \auxfile=\jobname.aux
\endgroup}


% Footnotes.

\newcount \footnoteno

\def\supereject{\par\penalty -20000\footnoteno =0 }

\let\ptexfootnote=\footnote

{\catcode `\@=11
\gdef\footnote{\global\advance \footnoteno by \@ne
\edef\thisfootno{$^{\the\footnoteno}$}%
\let\@sf\empty
\ifhmode\edef\@sf{\spacefactor\the\spacefactor}\/\fi
\thisfootno\@sf\parsearg\footnotezzz}

\gdef\footnotezzz #1{\insert\footins{
\interlinepenalty\interfootnotelinepenalty
\splittopskip\ht\strutbox % top baseline for broken footnotes
\splitmaxdepth\dp\strutbox \floatingpenalty\@MM
\leftskip\z@skip \rightskip\z@skip \spaceskip\z@skip \xspaceskip\z@skip
\footstrut\hang\textindent{\thisfootno}#1\strut}}

}%end \catcode `\@=11

% End of control word definitions.

\message{and turning on texinfo input format.}

\def\openindices{%
   \newindex{cp}%
   \newcodeindex{fn}%
   \newcodeindex{vr}%
   \newcodeindex{tp}%
   \newcodeindex{ky}%
   \newcodeindex{pg}%
}

% Set some numeric style parameters, for 8.5 x 11 format.

\hsize = 6.5in
\parindent 15pt
\parskip 18pt plus 1pt
\baselineskip 15pt
\advance\topskip by 1.2cm

% Prevent underfull vbox error messages.
\vbadness=10000

% Use @smallbook to reset parameters for 7x9.5 format
\def\smallbook{
\global\lispnarrowing = 0.3in
\global\baselineskip 12pt
\global\parskip 3pt plus 1pt
\global\hsize = 5in
\global\doublecolumnhsize=2.4in \global\doublecolumnvsize=15.0in
\global\vsize=7.5in
\global\tolerance=700
\global\hfuzz=1pt

\global\pagewidth=\hsize
\global\pageheight=\vsize
\global\font\ninett=cmtt9

\global\let\smalllisp=\smalllispx
\global\let\smallexample=\smalllispx
\global\def\Esmallexample{\Esmalllisp}
}

%% For a final copy, take out the rectangles
%% that mark overfull boxes (in case you have decided
%% that the text looks ok even though it passes the margin).
\def\finalout{\overfullrule=0pt}

% Turn off all special characters except @
% (and those which the user can use as if they were ordinary)
% Define certain chars to be always in tt font.

\catcode`\"=\active
\def\activedoublequote{{\tt \char '042}}
\let"=\activedoublequote
\catcode`\~=\active
\def~{{\tt \char '176}}
\chardef\hat=`\^
\catcode`\^=\active
\def^{{\tt \hat}}
\catcode`\_=\active
\def_{{\tt \char '137}}
\catcode`\|=\active
\def|{{\tt \char '174}}
\chardef \less=`\<
\catcode`\<=\active
\def<{{\tt \less}}
\chardef \gtr=`\>
\catcode`\>=\active
\def>{{\tt \gtr}}

\catcode`\@=0

% \rawbackslashxx output one backslash character in current font
{\catcode`\\=\other
@gdef@rawbackslashxx{\}}

% \rawbackslash redefines \ as input to do \rawbackslashxx.
{\catcode`\\=\active
@gdef@rawbackslash{@let\=@rawbackslashxx }}

% \normalbackslash outputs one backslash in fixed width font.
\def\normalbackslash{{\tt\rawbackslashxx}}

% Say @foo, not \foo, in error messages.
\escapechar=`\@

@c \catcode 17=0   @c Define control-q
\catcode`\\=\active

% If a .fmt file is being used, we don't want the `\input texinfo' to show up.
% That is what \eatinput is for; after that, the `\' should revert to printing 
% a backslash.
%
@gdef@eatinput input texinfo{@fixbackslash}
@global@let\ = @eatinput

% On the other hand, perhaps the file did not have a `\input texinfo'. Then
% the first `\{ in the file would cause an error. This macro tries to fix 
% that, assuming it is called before the first `\' could plausibly occur.
% 
@gdef@fixbackslash{@ifx\@eatinput @let\ = @normalbackslash @fi}

%% These look ok in all fonts, so just make them not special.  The @rm below
%% makes sure that the current font starts out as the newly loaded cmr10
@catcode`@$=@other @catcode`@%=@other @catcode`@&=@other @catcode`@#=@other

@textfonts
@rm
