%%% ====================================================================
%%%   This file is freely redistributable and placed into the
%%%   public domain by Tomas Rokicki.
%%%  @TeX-file{
%%%     author          = "Tom Rokicki",
%%%     version         = "2.7k",
%%%     date            = "19 July 1997",
%%%     time            = "10:00:05 MDT",
%%%     filename        = "epsf.tex",
%%%     address         = "Tom Rokicki
%%%                        Box 2081
%%%                        Stanford, CA 94309
%%%                        USA",
%%%     telephone       = "+1 415 855 9989",
%%%     email           = "rokicki@cs.stanford.edu (Internet)",
%%%     codetable       = "ISO/ASCII",
%%%     keywords        = "PostScript, TeX",
%%%     supported       = "yes",
%%%     abstract        = "This file contains macros to support the inclusion
%%%                        of Encapsulated PostScript files in TeX documents.",
%%%     docstring       = "This file contains TeX macros to include an
%%%                        Encapsulated PostScript graphic.  It works
%%%                        by finding the bounding box comment,
%%%                        calculating the correct scale values, and
%%%                        inserting a vbox of the appropriate size at
%%%                        the current position in the TeX document.
%%%
%%%                        To use, simply say
%%%
%%%                        \input epsf % somewhere early on in your TeX file
%%%
%%%                        % then where you want to insert a vbox for a figure:
%%%                        \epsfbox{filename.ps}
%%%
%%%                        Alternatively, you can supply your own
%%%                        bounding box by
%%%
%%%                        \epsfbox[0 0 30 50]{filename.ps}
%%%
%%%                        This will not read in the file, and will
%%%                        instead use the bounding box you specify.
%%%
%%%                        The effect will be to typeset the figure as
%%%                        a TeX box, at the point of your \epsfbox
%%%                        command. By default, the graphic will have
%%%                        its `natural' width (namely the width of
%%%                        its bounding box, as described in
%%%                        filename.ps). The TeX box will have depth
%%%                        zero.
%%%
%%%                        You can enlarge or reduce the figure by
%%%                        saying
%%%
%%%                          \epsfxsize=<dimen> \epsfbox{filename.ps}
%%%                        or
%%%                          \epsfysize=<dimen> \epsfbox{filename.ps}
%%%
%%%                        instead. Then the width of the TeX box will
%%%                        be \epsfxsize and its height will be scaled
%%%                        proportionately (or the height will be
%%%                        \epsfysize and its width will be scaled
%%%                        proportionately).
%%%
%%%                        The width (and height) is restored to zero
%%%                        after each use, so \epsfxsize or \epsfysize
%%%                        must be specified before EACH use of
%%%                        \epsfbox.
%%%
%%%                        A more general facility for sizing is
%%%                        available by defining the \epsfsize macro.
%%%                        Normally you can redefine this macro to do
%%%                        almost anything.  The first parameter is
%%%                        the natural x size of the PostScript
%%%                        graphic, the second parameter is the
%%%                        natural y size of the PostScript graphic.
%%%                        It must return the xsize to use, or 0 if
%%%                        natural scaling is to be used.  Common uses
%%%                        include:
%%%
%%%                           \epsfxsize  % just leave the old value alone
%%%                           0pt         % use the natural sizes
%%%                           #1          % use the natural sizes
%%%                           \hsize      % scale to full width
%%%                           0.5#1       % scale to 50% of natural size
%%%                           \ifnum #1>\hsize\hsize\else#1\fi
%%%                                       % smaller of natural, hsize
%%%
%%%                        If you want TeX to report the size of the
%%%                        figure (as a message on your terminal when
%%%                        it processes each figure), say
%%%                        `\epsfverbosetrue'.
%%%
%%%                        If you only want to get the bounding box
%%%                        extents, without producing any output boxes
%%%                        or \special{}, then say
%%%                        \epsfgetbb{filename}.  The extents will be
%%%                        saved in the macros \epsfllx \epsflly
%%%                        \epsfurx \epsfury in PostScript units of
%%%                        big points.
%%%
%%%                        Revision history:
%%%
%%%                        ---------------------------------------------
%%%                        epsf.tex macro file:
%%%                        Originally written by Tomas Rokicki of
%%%                        Radical Eye Software, 29 Mar 1989.
%%%
%%%                        ---------------------------------------------
%%%                        Revised by Don Knuth, 3 Jan 1990.
%%%
%%%                        ---------------------------------------------
%%%                        Revised by Tomas Rokicki, 18 Jul 1990.
%%%                        Accept bounding boxes with no space after
%%%                        the colon.
%%%
%%%                        ---------------------------------------------
%%%                        Revised by Nelson H. F. Beebe
%%%                        <beebe@math.utah.edu>, 03 Dec 1991 [2.0].
%%%                        Add version number and date typeout.
%%%
%%%                        Use \immediate\write16 instead of \message
%%%                        to ensure output on new line.
%%%
%%%                        Handle nested EPS files.
%%%
%%%                        Handle %%BoundingBox: (atend) lines.
%%%
%%%                        Do not quit when blank lines are found.
%%%
%%%                        Add a few percents to remove generation of
%%%                        spurious blank space.
%%%
%%%                        Move \special output to
%%%                        \epsfspecial{filename} so that other macro
%%%                        packages can input this one, then change
%%%                        the definition of \epsfspecial to match
%%%                        another DVI driver.
%%%
%%%                        Move size computation to \epsfsetsize which
%%%                        can be called by the user; the verbose
%%%                        output of the bounding box and scaled width
%%%                        and height happens here.
%%%
%%%                        ---------------------------------------------
%%%                        Revised by Nelson H. F. Beebe
%%%                        <beebe@math.utah.edu>, 05 May 1992 [2.1].
%%%                        Wrap \leavevmode\hbox{} around \vbox{} with
%%%                        the \special so that \epsffile{} can be
%%%                        used inside \begin{center}...\end{center}
%%%
%%%                        ---------------------------------------------
%%%                        Revised by Nelson H. F. Beebe
%%%                        <beebe@math.utah.edu>, 09 Dec 1992 [2.2].
%%%                        Introduce \epsfshow{true,false} and
%%%                        \epsfframe{true,false} macros; the latter
%%%                        suppresses the insertion of the PostScript,
%%%                        and instead just creates an empty box,
%%%                        which may be handy for rapid prototyping.
%%%
%%%                        ---------------------------------------------
%%%                        Revised by Nelson H. F. Beebe
%%%                        <beebe@math.utah.edu>, 14 Dec 1992 [2.3].
%%%                        Add \epsfshowfilename{true,false}.  When
%%%                        true, and \epsfshowfalse is specified, the
%%%                        PostScript file name will be displayed
%%%                        centered in the figure box.
%%%
%%%                        ---------------------------------------------
%%%                        Revised by Nelson H. F. Beebe
%%%                        <beebe@math.utah.edu>, 20 June 1993 [2.4].
%%%                        Remove non-zero debug setting of \epsfframemargin,
%%%                        and change margin handling to preserve EPS image
%%%                        size and aspect ratio, so that the actual
%%%                        box is \epsfxsize+\epsfframemargin wide by
%%%                        \epsfysize+\epsfframemargin high.
%%%                        Reduce output of \epsfshowfilenametrue to
%%%                        just the bare file name.
%%%
%%%                        ---------------------------------------------
%%%                        Revised by Nelson H. F. Beebe
%%%                        <beebe@math.utah.edu>, 13 July 1993 [2.5].
%%%                        Add \epsfframethickness for control of
%%%                        \epsfframe frame lines.
%%%
%%%                        ---------------------------------------------
%%%                        Revised by Nelson H. F. Beebe
%%%                        <beebe@math.utah.edu>, 02 July 1996 [2.6]
%%%                        Add missing initialization \epsfatendfalse;
%%%                        the lack of this resulted in the wrong
%%%                        BoundingBox being picked up, mea culpa, sigh...
%%%                        ---------------------------------------------
%%%
%%%                        ---------------------------------------------
%%%                        Revised by Nelson H. F. Beebe
%%%                        <beebe@math.utah.edu>, 25 October 1996 [2.7]
%%%                        Update to match changes in from dvips 5-600
%%%                        distribution: new user-accessible macros:
%%%                        \epsfclipon, \epsfclipoff, \epsfdrafton,
%%%                        \epsfdraftoff, change \empty to \epsfempty.
%%%                        ---------------------------------------------
%%%                        
%%%                        Modified to avoid verbosity, give help.
%%%                        --kb@cs.umb.edu, for Texinfo.
%%%  }
%%% ====================================================================
%
\ifx\epsfannounce\undefined \def\epsfannounce{\immediate\write16}\fi
 \epsfannounce{This is `epsf.tex' v2.7k <10 July 1997>}%
%
\newread\epsffilein    % file to \read
\newif\ifepsfatend     % need to scan to LAST %%BoundingBox comment?
\newif\ifepsfbbfound   % success?
\newif\ifepsfdraft     % use draft mode?
\newif\ifepsffileok    % continue looking for the bounding box?
\newif\ifepsfframe     % frame the bounding box?
\newif\ifepsfshow      % show PostScript file, or just bounding box?
\epsfshowtrue          % default is to display PostScript file
\newif\ifepsfshowfilename % show the file name if \epsfshowfalse specified?
\newif\ifepsfverbose   % report what you're making?
\newdimen\epsfframemargin % margin between box and frame
\newdimen\epsfframethickness % thickness of frame rules
\newdimen\epsfrsize    % vertical size before scaling
\newdimen\epsftmp      % register for arithmetic manipulation
\newdimen\epsftsize    % horizontal size before scaling
\newdimen\epsfxsize    % horizontal size after scaling
\newdimen\epsfysize    % vertical size after scaling
\newdimen\pspoints     % conversion factor
%
\pspoints = 1bp        % Adobe points are `big'
\epsfxsize = 0pt       % default value, means `use natural size'
\epsfysize = 0pt       % ditto
\epsfframemargin = 0pt % default value: frame box flush around picture
\epsfframethickness = 0.4pt % TeX's default rule thickness
%
\def\epsfbox#1{\global\def\epsfllx{72}\global\def\epsflly{72}%
   \global\def\epsfurx{540}\global\def\epsfury{720}%
   \def\lbracket{[}\def\testit{#1}\ifx\testit\lbracket
   \let\next=\epsfgetlitbb\else\let\next=\epsfnormal\fi\next{#1}}%
%
% We use \epsfgetlitbb if the user specified an explicit bounding box,
% and \epsfnormal otherwise.  Because \epsfgetbb can be called
% separately to retrieve the bounding box, we move the verbose
% printing the bounding box extents and size on the terminal to
% \epsfstatus.  Therefore, when the user provided the bounding box,
% \epsfgetbb will not be called, so we must call \epsfsetsize and
% \epsfstatus ourselves.
%
\def\epsfgetlitbb#1#2 #3 #4 #5]#6{%
   \epsfgrab #2 #3 #4 #5 .\\%
   \epsfsetsize
   \epsfstatus{#6}%
   \epsfsetgraph{#6}%
}%
%
\def\epsfnormal#1{%
    \epsfgetbb{#1}%
    \epsfsetgraph{#1}%
}%
%
\newhelp\epsfnoopenhelp{The PostScript image file must be findable by
TeX, i.e., somewhere in the TEXINPUTS (or equivalent) path.}%
%
\def\epsfgetbb#1{%
%
%   The first thing we need to do is to open the
%   PostScript file, if possible.
%
    \openin\epsffilein=#1
    \ifeof\epsffilein
        \errhelp = \epsfnoopenhelp
        \errmessage{Could not open file #1, ignoring it}%
    \else                       %process the file
        {%                      %start a group to contain catcode changes
            % Make all special characters, except space, to be of type
            % `other' so we process the file in almost verbatim mode
            % (TeXbook, p. 344).
            \chardef\other=12
            \def\do##1{\catcode`##1=\other}%
            \dospecials
            \catcode`\ =10
            \epsffileoktrue         %true while we are looping
            \epsfatendfalse     %[02-Jul-1996]: add forgotten initialization
            \loop               %reading lines from the EPS file
                \read\epsffilein to \epsffileline
                \ifeof\epsffilein %then no more input
                \epsffileokfalse %so set completion flag
            \else                %otherwise process one line
                \expandafter\epsfaux\epsffileline:. \\%
            \fi
            \ifepsffileok
            \repeat
            \ifepsfbbfound
            \else
                \ifepsfverbose
                    \immediate\write16{No BoundingBox comment found in %
                                    file #1; using defaults}%
                \fi
            \fi
        }%                      %end catcode changes
        \closein\epsffilein
    \fi                         %end of file processing
    \epsfsetsize                %compute size parameters
    \epsfstatus{#1}%
}%
%
% Clipping control:
\def\epsfclipon{\def\epsfclipstring{ clip}}%
\def\epsfclipoff{\def\epsfclipstring{\ifepsfdraft\space clip\fi}}%
\epsfclipoff % default for dvips is OFF
%
% The special that is emitted by \epsfsetgraph comes from this macro.
% It is defined separately to allow easy customization by other
% packages that first \input epsf.tex, then redefine \epsfspecial.
% This macro is invoked in the lower-left corner of a box of the
% width and height determined from the arguments to \epsffile, or
% from the %%BoundingBox in the EPS file itself.
%
% This version is for dvips:
\def\epsfspecial#1{%
     \epsftmp=10\epsfxsize
     \divide\epsftmp\pspoints
     \ifnum\epsfrsize=0\relax
       \special{PSfile=\ifepsfdraft psdraft.ps\else#1\fi\space
                llx=\epsfllx\space
                lly=\epsflly\space
                urx=\epsfurx\space
                ury=\epsfury\space
                rwi=\number\epsftmp
                \epsfclipstring
               }%
     \else
       \epsfrsize=10\epsfysize
       \divide\epsfrsize\pspoints
       \special{PSfile=\ifepsfdraft psdraft.ps\else#1\fi\space
                llx=\epsfllx\space
                lly=\epsflly\space
                urx=\epsfurx\space
                ury=\epsfury\space
                rwi=\number\epsftmp\space
                rhi=\number\epsfrsize
                \epsfclipstring
               }%
     \fi
}%
%
% \epsfframe macro adapted from the TeXbook, exercise 21.3, p. 223, 331.
% but modified to set the box width to the natural width, rather
% than the line width, and to include space for margins and rules
\def\epsfframe#1%
{%
  \leavevmode                   % so we can put this inside
                                % a centered environment
  \setbox0 = \hbox{#1}%
  \dimen0 = \wd0                                % natural width of argument
  \advance \dimen0 by 2\epsfframemargin         % plus width of 2 margins
  \advance \dimen0 by 2\epsfframethickness      % plus width of 2 rule lines
  \vbox
  {%
    \hrule height \epsfframethickness depth 0pt
    \hbox to \dimen0
    {%
      \hss
      \vrule width \epsfframethickness
      \kern \epsfframemargin
      \vbox {\kern \epsfframemargin \box0 \kern \epsfframemargin }%
      \kern \epsfframemargin
      \vrule width \epsfframethickness
      \hss
    }% end hbox
    \hrule height 0pt depth \epsfframethickness
  }% end vbox
}%
%
\def\epsfsetgraph#1%
{%
   %
   % Make the vbox and stick in a \special that the DVI driver can
   % parse.  \vfil and \hfil are used to place the \special origin at
   % the lower-left corner of the vbox.  \epsfspecial can be redefined
   % to produce alternate \special syntaxes.
   %
   \leavevmode
   \hbox{% so we can put this in \begin{center}...\end{center}
     \ifepsfframe\expandafter\epsfframe\fi
     {\vbox to\epsfysize
     {%
        \ifepsfshow
            % output \special{} at lower-left corner of figure box
            \vfil
            \hbox to \epsfxsize{\epsfspecial{#1}\hfil}%
        \else
            \vfil
            \hbox to\epsfxsize{%
               \hss
               \ifepsfshowfilename
               {%
                  \epsfframemargin=3pt % local change of margin
                  \epsfframe{{\tt #1}}%
               }%
               \fi
               \hss
            }%
            \vfil
        \fi
     }%
   }}%
   %
   % Reset \epsfxsize and \epsfysize, as documented above.
   %
   \global\epsfxsize=0pt
   \global\epsfysize=0pt
}%
%
%   Now we have to calculate the scale and offset values to use.
%   First we compute the natural sizes.
%
\def\epsfsetsize
{%
   \epsfrsize=\epsfury\pspoints
   \advance\epsfrsize by-\epsflly\pspoints
   \epsftsize=\epsfurx\pspoints
   \advance\epsftsize by-\epsfllx\pspoints
%
%   If `epsfxsize' is 0, we default to the natural size of the picture.
%   Otherwise we scale the graph to be \epsfxsize wide.
%
   \epsfxsize=\epsfsize{\epsftsize}{\epsfrsize}%
   \ifnum \epsfxsize=0
      \ifnum \epsfysize=0
        \epsfxsize=\epsftsize
        \epsfysize=\epsfrsize
        \epsfrsize=0pt
%
%   We have a sticky problem here:  TeX doesn't do floating point arithmetic!
%   Our goal is to compute y = rx/t. The following loop does this reasonably
%   fast, with an error of at most about 16 sp (about 1/4000 pt).
%
      \else
        \epsftmp=\epsftsize \divide\epsftmp\epsfrsize
        \epsfxsize=\epsfysize \multiply\epsfxsize\epsftmp
        \multiply\epsftmp\epsfrsize \advance\epsftsize-\epsftmp
        \epsftmp=\epsfysize
        \loop \advance\epsftsize\epsftsize \divide\epsftmp 2
        \ifnum \epsftmp>0
           \ifnum \epsftsize<\epsfrsize
           \else
              \advance\epsftsize-\epsfrsize \advance\epsfxsize\epsftmp
           \fi
        \repeat
        \epsfrsize=0pt
      \fi
   \else
     \ifnum \epsfysize=0
       \epsftmp=\epsfrsize \divide\epsftmp\epsftsize
       \epsfysize=\epsfxsize \multiply\epsfysize\epsftmp
       \multiply\epsftmp\epsftsize \advance\epsfrsize-\epsftmp
       \epsftmp=\epsfxsize
       \loop \advance\epsfrsize\epsfrsize \divide\epsftmp 2
       \ifnum \epsftmp>0
          \ifnum \epsfrsize<\epsftsize
          \else
             \advance\epsfrsize-\epsftsize \advance\epsfysize\epsftmp
          \fi
       \repeat
       \epsfrsize=0pt
     \else
       \epsfrsize=\epsfysize
     \fi
   \fi
}%
%
% Issue some status messages if the user requested them
%
\def\epsfstatus#1{% arg = filename
   \ifepsfverbose
     \immediate\write16{#1: BoundingBox:
                  llx = \epsfllx\space lly = \epsflly\space
                  urx = \epsfurx\space ury = \epsfury\space}%
     \immediate\write16{#1: scaled width = \the\epsfxsize\space
                  scaled height = \the\epsfysize}%
   \fi
}%
%
%   We still need to define the tricky \epsfaux macro. This requires
%   a couple of magic constants for comparison purposes.
%
{\catcode`\%=12 \global\let\epsfpercent=%\global\def\epsfbblit{%BoundingBox}}%
\global\def\epsfatend{(atend)}%
%
%   So we're ready to check for `%BoundingBox:' and to grab the
%   values if they are found.
%
%   If we find a line
%
%   %%BoundingBox: (atend)
%
%   then we ignore it, but set a flag to force parsing all of the
%   file, so the last %%BoundingBox parsed will be the one used.  This
%   is necessary, because EPS files can themselves contain other EPS
%   files with their own %%BoundingBox comments.
%
%   If we find a line
%
%   %%BoundingBox: llx lly urx ury
%
%   then we save the 4 values in \epsfllx, \epsflly, \epsfurx, \epsfury.
%   Then, if we have not previously parsed an (atend), we flag completion
%   and can stop reading the file.  Otherwise, we must keep on reading
%   to end of file so that we find the values on the LAST %%BoundingBox.
\long\def\epsfaux#1#2:#3\\%
{%
   \def\testit{#2}%             % save second character up to just before colon
   \ifx#1\epsfpercent           % then first char is percent (quick test)
       \ifx\testit\epsfbblit    % then (slow test) we have %%BoundingBox
            \epsfgrab #3 . . . \\%
            \ifx\epsfllx\epsfatend % then ignore %%BoundingBox: (atend)
                \global\epsfatendtrue
            \else               % else found %%BoundingBox: llx lly urx ury
                \ifepsfatend    % then keep parsing ALL %%BoundingBox lines
                \else           % else stop after first one parsed
                    \epsffileokfalse
                \fi
                \global\epsfbbfoundtrue
            \fi
       \fi
   \fi
}%
%
%   Here we grab the values and stuff them in the appropriate definitions.
%
\def\epsfempty{}%
\def\epsfgrab #1 #2 #3 #4 #5\\{%
   \global\def\epsfllx{#1}\ifx\epsfllx\epsfempty
      \epsfgrab #2 #3 #4 #5 .\\\else
   \global\def\epsflly{#2}%
   \global\def\epsfurx{#3}\global\def\epsfury{#4}\fi
}%
%
%   We default the epsfsize macro.
%
\def\epsfsize#1#2{\epsfxsize}%
%
%   Finally, another definition for compatibility with older macros.
%
\let\epsffile=\epsfbox
\endinput

